\documentclass[11pt]{article}

    \usepackage[breakable]{tcolorbox}
    \usepackage{parskip} % Stop auto-indenting (to mimic markdown behaviour)
    
    \usepackage{iftex}
    \ifPDFTeX
    	\usepackage[T1]{fontenc}
    	\usepackage{mathpazo}
    \else
    	\usepackage{fontspec}
    \fi

    % Basic figure setup, for now with no caption control since it's done
    % automatically by Pandoc (which extracts ![](path) syntax from Markdown).
    \usepackage{graphicx}
    % Maintain compatibility with old templates. Remove in nbconvert 6.0
    \let\Oldincludegraphics\includegraphics
    % Ensure that by default, figures have no caption (until we provide a
    % proper Figure object with a Caption API and a way to capture that
    % in the conversion process - todo).
    \usepackage{caption}
    \DeclareCaptionFormat{nocaption}{}
    \captionsetup{format=nocaption,aboveskip=0pt,belowskip=0pt}

    \usepackage[Export]{adjustbox} % Used to constrain images to a maximum size
    \adjustboxset{max size={0.9\linewidth}{0.9\paperheight}}
    \usepackage{float}
    \floatplacement{figure}{H} % forces figures to be placed at the correct location
    \usepackage{xcolor} % Allow colors to be defined
    \usepackage{enumerate} % Needed for markdown enumerations to work
    \usepackage{geometry} % Used to adjust the document margins
    \usepackage{amsmath} % Equations
    \usepackage{amssymb} % Equations
    \usepackage{textcomp} % defines textquotesingle
    % Hack from http://tex.stackexchange.com/a/47451/13684:
    \AtBeginDocument{%
        \def\PYZsq{\textquotesingle}% Upright quotes in Pygmentized code
    }
    \usepackage{upquote} % Upright quotes for verbatim code
    \usepackage{eurosym} % defines \euro
    \usepackage[mathletters]{ucs} % Extended unicode (utf-8) support
    \usepackage{fancyvrb} % verbatim replacement that allows latex
    \usepackage{grffile} % extends the file name processing of package graphics 
                         % to support a larger range
    \makeatletter % fix for grffile with XeLaTeX
    \def\Gread@@xetex#1{%
      \IfFileExists{"\Gin@base".bb}%
      {\Gread@eps{\Gin@base.bb}}%
      {\Gread@@xetex@aux#1}%
    }
    \makeatother

    % The hyperref package gives us a pdf with properly built
    % internal navigation ('pdf bookmarks' for the table of contents,
    % internal cross-reference links, web links for URLs, etc.)
    \usepackage{hyperref}
    % The default LaTeX title has an obnoxious amount of whitespace. By default,
    % titling removes some of it. It also provides customization options.
    \usepackage{titling}
    \usepackage{longtable} % longtable support required by pandoc >1.10
    \usepackage{booktabs}  % table support for pandoc > 1.12.2
    \usepackage[inline]{enumitem} % IRkernel/repr support (it uses the enumerate* environment)
    \usepackage[normalem]{ulem} % ulem is needed to support strikethroughs (\sout)
                                % normalem makes italics be italics, not underlines
    \usepackage{mathrsfs}
    

    
    % Colors for the hyperref package
    \definecolor{urlcolor}{rgb}{0,.145,.698}
    \definecolor{linkcolor}{rgb}{.71,0.21,0.01}
    \definecolor{citecolor}{rgb}{.12,.54,.11}

    % ANSI colors
    \definecolor{ansi-black}{HTML}{3E424D}
    \definecolor{ansi-black-intense}{HTML}{282C36}
    \definecolor{ansi-red}{HTML}{E75C58}
    \definecolor{ansi-red-intense}{HTML}{B22B31}
    \definecolor{ansi-green}{HTML}{00A250}
    \definecolor{ansi-green-intense}{HTML}{007427}
    \definecolor{ansi-yellow}{HTML}{DDB62B}
    \definecolor{ansi-yellow-intense}{HTML}{B27D12}
    \definecolor{ansi-blue}{HTML}{208FFB}
    \definecolor{ansi-blue-intense}{HTML}{0065CA}
    \definecolor{ansi-magenta}{HTML}{D160C4}
    \definecolor{ansi-magenta-intense}{HTML}{A03196}
    \definecolor{ansi-cyan}{HTML}{60C6C8}
    \definecolor{ansi-cyan-intense}{HTML}{258F8F}
    \definecolor{ansi-white}{HTML}{C5C1B4}
    \definecolor{ansi-white-intense}{HTML}{A1A6B2}
    \definecolor{ansi-default-inverse-fg}{HTML}{FFFFFF}
    \definecolor{ansi-default-inverse-bg}{HTML}{000000}

    % commands and environments needed by pandoc snippets
    % extracted from the output of `pandoc -s`
    \providecommand{\tightlist}{%
      \setlength{\itemsep}{0pt}\setlength{\parskip}{0pt}}
    \DefineVerbatimEnvironment{Highlighting}{Verbatim}{commandchars=\\\{\}}
    % Add ',fontsize=\small' for more characters per line
    \newenvironment{Shaded}{}{}
    \newcommand{\KeywordTok}[1]{\textcolor[rgb]{0.00,0.44,0.13}{\textbf{{#1}}}}
    \newcommand{\DataTypeTok}[1]{\textcolor[rgb]{0.56,0.13,0.00}{{#1}}}
    \newcommand{\DecValTok}[1]{\textcolor[rgb]{0.25,0.63,0.44}{{#1}}}
    \newcommand{\BaseNTok}[1]{\textcolor[rgb]{0.25,0.63,0.44}{{#1}}}
    \newcommand{\FloatTok}[1]{\textcolor[rgb]{0.25,0.63,0.44}{{#1}}}
    \newcommand{\CharTok}[1]{\textcolor[rgb]{0.25,0.44,0.63}{{#1}}}
    \newcommand{\StringTok}[1]{\textcolor[rgb]{0.25,0.44,0.63}{{#1}}}
    \newcommand{\CommentTok}[1]{\textcolor[rgb]{0.38,0.63,0.69}{\textit{{#1}}}}
    \newcommand{\OtherTok}[1]{\textcolor[rgb]{0.00,0.44,0.13}{{#1}}}
    \newcommand{\AlertTok}[1]{\textcolor[rgb]{1.00,0.00,0.00}{\textbf{{#1}}}}
    \newcommand{\FunctionTok}[1]{\textcolor[rgb]{0.02,0.16,0.49}{{#1}}}
    \newcommand{\RegionMarkerTok}[1]{{#1}}
    \newcommand{\ErrorTok}[1]{\textcolor[rgb]{1.00,0.00,0.00}{\textbf{{#1}}}}
    \newcommand{\NormalTok}[1]{{#1}}
    
    % Additional commands for more recent versions of Pandoc
    \newcommand{\ConstantTok}[1]{\textcolor[rgb]{0.53,0.00,0.00}{{#1}}}
    \newcommand{\SpecialCharTok}[1]{\textcolor[rgb]{0.25,0.44,0.63}{{#1}}}
    \newcommand{\VerbatimStringTok}[1]{\textcolor[rgb]{0.25,0.44,0.63}{{#1}}}
    \newcommand{\SpecialStringTok}[1]{\textcolor[rgb]{0.73,0.40,0.53}{{#1}}}
    \newcommand{\ImportTok}[1]{{#1}}
    \newcommand{\DocumentationTok}[1]{\textcolor[rgb]{0.73,0.13,0.13}{\textit{{#1}}}}
    \newcommand{\AnnotationTok}[1]{\textcolor[rgb]{0.38,0.63,0.69}{\textbf{\textit{{#1}}}}}
    \newcommand{\CommentVarTok}[1]{\textcolor[rgb]{0.38,0.63,0.69}{\textbf{\textit{{#1}}}}}
    \newcommand{\VariableTok}[1]{\textcolor[rgb]{0.10,0.09,0.49}{{#1}}}
    \newcommand{\ControlFlowTok}[1]{\textcolor[rgb]{0.00,0.44,0.13}{\textbf{{#1}}}}
    \newcommand{\OperatorTok}[1]{\textcolor[rgb]{0.40,0.40,0.40}{{#1}}}
    \newcommand{\BuiltInTok}[1]{{#1}}
    \newcommand{\ExtensionTok}[1]{{#1}}
    \newcommand{\PreprocessorTok}[1]{\textcolor[rgb]{0.74,0.48,0.00}{{#1}}}
    \newcommand{\AttributeTok}[1]{\textcolor[rgb]{0.49,0.56,0.16}{{#1}}}
    \newcommand{\InformationTok}[1]{\textcolor[rgb]{0.38,0.63,0.69}{\textbf{\textit{{#1}}}}}
    \newcommand{\WarningTok}[1]{\textcolor[rgb]{0.38,0.63,0.69}{\textbf{\textit{{#1}}}}}
    
    
    % Define a nice break command that doesn't care if a line doesn't already
    % exist.
    \def\br{\hspace*{\fill} \\* }
    % Math Jax compatibility definitions
    \def\gt{>}
    \def\lt{<}
    \let\Oldtex\TeX
    \let\Oldlatex\LaTeX
    \renewcommand{\TeX}{\textrm{\Oldtex}}
    \renewcommand{\LaTeX}{\textrm{\Oldlatex}}
    % Document parameters
    % Document title
    \title{Face\_Recognition}
    
    
    
    
    
% Pygments definitions
\makeatletter
\def\PY@reset{\let\PY@it=\relax \let\PY@bf=\relax%
    \let\PY@ul=\relax \let\PY@tc=\relax%
    \let\PY@bc=\relax \let\PY@ff=\relax}
\def\PY@tok#1{\csname PY@tok@#1\endcsname}
\def\PY@toks#1+{\ifx\relax#1\empty\else%
    \PY@tok{#1}\expandafter\PY@toks\fi}
\def\PY@do#1{\PY@bc{\PY@tc{\PY@ul{%
    \PY@it{\PY@bf{\PY@ff{#1}}}}}}}
\def\PY#1#2{\PY@reset\PY@toks#1+\relax+\PY@do{#2}}

\expandafter\def\csname PY@tok@w\endcsname{\def\PY@tc##1{\textcolor[rgb]{0.73,0.73,0.73}{##1}}}
\expandafter\def\csname PY@tok@c\endcsname{\let\PY@it=\textit\def\PY@tc##1{\textcolor[rgb]{0.25,0.50,0.50}{##1}}}
\expandafter\def\csname PY@tok@cp\endcsname{\def\PY@tc##1{\textcolor[rgb]{0.74,0.48,0.00}{##1}}}
\expandafter\def\csname PY@tok@k\endcsname{\let\PY@bf=\textbf\def\PY@tc##1{\textcolor[rgb]{0.00,0.50,0.00}{##1}}}
\expandafter\def\csname PY@tok@kp\endcsname{\def\PY@tc##1{\textcolor[rgb]{0.00,0.50,0.00}{##1}}}
\expandafter\def\csname PY@tok@kt\endcsname{\def\PY@tc##1{\textcolor[rgb]{0.69,0.00,0.25}{##1}}}
\expandafter\def\csname PY@tok@o\endcsname{\def\PY@tc##1{\textcolor[rgb]{0.40,0.40,0.40}{##1}}}
\expandafter\def\csname PY@tok@ow\endcsname{\let\PY@bf=\textbf\def\PY@tc##1{\textcolor[rgb]{0.67,0.13,1.00}{##1}}}
\expandafter\def\csname PY@tok@nb\endcsname{\def\PY@tc##1{\textcolor[rgb]{0.00,0.50,0.00}{##1}}}
\expandafter\def\csname PY@tok@nf\endcsname{\def\PY@tc##1{\textcolor[rgb]{0.00,0.00,1.00}{##1}}}
\expandafter\def\csname PY@tok@nc\endcsname{\let\PY@bf=\textbf\def\PY@tc##1{\textcolor[rgb]{0.00,0.00,1.00}{##1}}}
\expandafter\def\csname PY@tok@nn\endcsname{\let\PY@bf=\textbf\def\PY@tc##1{\textcolor[rgb]{0.00,0.00,1.00}{##1}}}
\expandafter\def\csname PY@tok@ne\endcsname{\let\PY@bf=\textbf\def\PY@tc##1{\textcolor[rgb]{0.82,0.25,0.23}{##1}}}
\expandafter\def\csname PY@tok@nv\endcsname{\def\PY@tc##1{\textcolor[rgb]{0.10,0.09,0.49}{##1}}}
\expandafter\def\csname PY@tok@no\endcsname{\def\PY@tc##1{\textcolor[rgb]{0.53,0.00,0.00}{##1}}}
\expandafter\def\csname PY@tok@nl\endcsname{\def\PY@tc##1{\textcolor[rgb]{0.63,0.63,0.00}{##1}}}
\expandafter\def\csname PY@tok@ni\endcsname{\let\PY@bf=\textbf\def\PY@tc##1{\textcolor[rgb]{0.60,0.60,0.60}{##1}}}
\expandafter\def\csname PY@tok@na\endcsname{\def\PY@tc##1{\textcolor[rgb]{0.49,0.56,0.16}{##1}}}
\expandafter\def\csname PY@tok@nt\endcsname{\let\PY@bf=\textbf\def\PY@tc##1{\textcolor[rgb]{0.00,0.50,0.00}{##1}}}
\expandafter\def\csname PY@tok@nd\endcsname{\def\PY@tc##1{\textcolor[rgb]{0.67,0.13,1.00}{##1}}}
\expandafter\def\csname PY@tok@s\endcsname{\def\PY@tc##1{\textcolor[rgb]{0.73,0.13,0.13}{##1}}}
\expandafter\def\csname PY@tok@sd\endcsname{\let\PY@it=\textit\def\PY@tc##1{\textcolor[rgb]{0.73,0.13,0.13}{##1}}}
\expandafter\def\csname PY@tok@si\endcsname{\let\PY@bf=\textbf\def\PY@tc##1{\textcolor[rgb]{0.73,0.40,0.53}{##1}}}
\expandafter\def\csname PY@tok@se\endcsname{\let\PY@bf=\textbf\def\PY@tc##1{\textcolor[rgb]{0.73,0.40,0.13}{##1}}}
\expandafter\def\csname PY@tok@sr\endcsname{\def\PY@tc##1{\textcolor[rgb]{0.73,0.40,0.53}{##1}}}
\expandafter\def\csname PY@tok@ss\endcsname{\def\PY@tc##1{\textcolor[rgb]{0.10,0.09,0.49}{##1}}}
\expandafter\def\csname PY@tok@sx\endcsname{\def\PY@tc##1{\textcolor[rgb]{0.00,0.50,0.00}{##1}}}
\expandafter\def\csname PY@tok@m\endcsname{\def\PY@tc##1{\textcolor[rgb]{0.40,0.40,0.40}{##1}}}
\expandafter\def\csname PY@tok@gh\endcsname{\let\PY@bf=\textbf\def\PY@tc##1{\textcolor[rgb]{0.00,0.00,0.50}{##1}}}
\expandafter\def\csname PY@tok@gu\endcsname{\let\PY@bf=\textbf\def\PY@tc##1{\textcolor[rgb]{0.50,0.00,0.50}{##1}}}
\expandafter\def\csname PY@tok@gd\endcsname{\def\PY@tc##1{\textcolor[rgb]{0.63,0.00,0.00}{##1}}}
\expandafter\def\csname PY@tok@gi\endcsname{\def\PY@tc##1{\textcolor[rgb]{0.00,0.63,0.00}{##1}}}
\expandafter\def\csname PY@tok@gr\endcsname{\def\PY@tc##1{\textcolor[rgb]{1.00,0.00,0.00}{##1}}}
\expandafter\def\csname PY@tok@ge\endcsname{\let\PY@it=\textit}
\expandafter\def\csname PY@tok@gs\endcsname{\let\PY@bf=\textbf}
\expandafter\def\csname PY@tok@gp\endcsname{\let\PY@bf=\textbf\def\PY@tc##1{\textcolor[rgb]{0.00,0.00,0.50}{##1}}}
\expandafter\def\csname PY@tok@go\endcsname{\def\PY@tc##1{\textcolor[rgb]{0.53,0.53,0.53}{##1}}}
\expandafter\def\csname PY@tok@gt\endcsname{\def\PY@tc##1{\textcolor[rgb]{0.00,0.27,0.87}{##1}}}
\expandafter\def\csname PY@tok@err\endcsname{\def\PY@bc##1{\setlength{\fboxsep}{0pt}\fcolorbox[rgb]{1.00,0.00,0.00}{1,1,1}{\strut ##1}}}
\expandafter\def\csname PY@tok@kc\endcsname{\let\PY@bf=\textbf\def\PY@tc##1{\textcolor[rgb]{0.00,0.50,0.00}{##1}}}
\expandafter\def\csname PY@tok@kd\endcsname{\let\PY@bf=\textbf\def\PY@tc##1{\textcolor[rgb]{0.00,0.50,0.00}{##1}}}
\expandafter\def\csname PY@tok@kn\endcsname{\let\PY@bf=\textbf\def\PY@tc##1{\textcolor[rgb]{0.00,0.50,0.00}{##1}}}
\expandafter\def\csname PY@tok@kr\endcsname{\let\PY@bf=\textbf\def\PY@tc##1{\textcolor[rgb]{0.00,0.50,0.00}{##1}}}
\expandafter\def\csname PY@tok@bp\endcsname{\def\PY@tc##1{\textcolor[rgb]{0.00,0.50,0.00}{##1}}}
\expandafter\def\csname PY@tok@fm\endcsname{\def\PY@tc##1{\textcolor[rgb]{0.00,0.00,1.00}{##1}}}
\expandafter\def\csname PY@tok@vc\endcsname{\def\PY@tc##1{\textcolor[rgb]{0.10,0.09,0.49}{##1}}}
\expandafter\def\csname PY@tok@vg\endcsname{\def\PY@tc##1{\textcolor[rgb]{0.10,0.09,0.49}{##1}}}
\expandafter\def\csname PY@tok@vi\endcsname{\def\PY@tc##1{\textcolor[rgb]{0.10,0.09,0.49}{##1}}}
\expandafter\def\csname PY@tok@vm\endcsname{\def\PY@tc##1{\textcolor[rgb]{0.10,0.09,0.49}{##1}}}
\expandafter\def\csname PY@tok@sa\endcsname{\def\PY@tc##1{\textcolor[rgb]{0.73,0.13,0.13}{##1}}}
\expandafter\def\csname PY@tok@sb\endcsname{\def\PY@tc##1{\textcolor[rgb]{0.73,0.13,0.13}{##1}}}
\expandafter\def\csname PY@tok@sc\endcsname{\def\PY@tc##1{\textcolor[rgb]{0.73,0.13,0.13}{##1}}}
\expandafter\def\csname PY@tok@dl\endcsname{\def\PY@tc##1{\textcolor[rgb]{0.73,0.13,0.13}{##1}}}
\expandafter\def\csname PY@tok@s2\endcsname{\def\PY@tc##1{\textcolor[rgb]{0.73,0.13,0.13}{##1}}}
\expandafter\def\csname PY@tok@sh\endcsname{\def\PY@tc##1{\textcolor[rgb]{0.73,0.13,0.13}{##1}}}
\expandafter\def\csname PY@tok@s1\endcsname{\def\PY@tc##1{\textcolor[rgb]{0.73,0.13,0.13}{##1}}}
\expandafter\def\csname PY@tok@mb\endcsname{\def\PY@tc##1{\textcolor[rgb]{0.40,0.40,0.40}{##1}}}
\expandafter\def\csname PY@tok@mf\endcsname{\def\PY@tc##1{\textcolor[rgb]{0.40,0.40,0.40}{##1}}}
\expandafter\def\csname PY@tok@mh\endcsname{\def\PY@tc##1{\textcolor[rgb]{0.40,0.40,0.40}{##1}}}
\expandafter\def\csname PY@tok@mi\endcsname{\def\PY@tc##1{\textcolor[rgb]{0.40,0.40,0.40}{##1}}}
\expandafter\def\csname PY@tok@il\endcsname{\def\PY@tc##1{\textcolor[rgb]{0.40,0.40,0.40}{##1}}}
\expandafter\def\csname PY@tok@mo\endcsname{\def\PY@tc##1{\textcolor[rgb]{0.40,0.40,0.40}{##1}}}
\expandafter\def\csname PY@tok@ch\endcsname{\let\PY@it=\textit\def\PY@tc##1{\textcolor[rgb]{0.25,0.50,0.50}{##1}}}
\expandafter\def\csname PY@tok@cm\endcsname{\let\PY@it=\textit\def\PY@tc##1{\textcolor[rgb]{0.25,0.50,0.50}{##1}}}
\expandafter\def\csname PY@tok@cpf\endcsname{\let\PY@it=\textit\def\PY@tc##1{\textcolor[rgb]{0.25,0.50,0.50}{##1}}}
\expandafter\def\csname PY@tok@c1\endcsname{\let\PY@it=\textit\def\PY@tc##1{\textcolor[rgb]{0.25,0.50,0.50}{##1}}}
\expandafter\def\csname PY@tok@cs\endcsname{\let\PY@it=\textit\def\PY@tc##1{\textcolor[rgb]{0.25,0.50,0.50}{##1}}}

\def\PYZbs{\char`\\}
\def\PYZus{\char`\_}
\def\PYZob{\char`\{}
\def\PYZcb{\char`\}}
\def\PYZca{\char`\^}
\def\PYZam{\char`\&}
\def\PYZlt{\char`\<}
\def\PYZgt{\char`\>}
\def\PYZsh{\char`\#}
\def\PYZpc{\char`\%}
\def\PYZdl{\char`\$}
\def\PYZhy{\char`\-}
\def\PYZsq{\char`\'}
\def\PYZdq{\char`\"}
\def\PYZti{\char`\~}
% for compatibility with earlier versions
\def\PYZat{@}
\def\PYZlb{[}
\def\PYZrb{]}
\makeatother


    % For linebreaks inside Verbatim environment from package fancyvrb. 
    \makeatletter
        \newbox\Wrappedcontinuationbox 
        \newbox\Wrappedvisiblespacebox 
        \newcommand*\Wrappedvisiblespace {\textcolor{red}{\textvisiblespace}} 
        \newcommand*\Wrappedcontinuationsymbol {\textcolor{red}{\llap{\tiny$\m@th\hookrightarrow$}}} 
        \newcommand*\Wrappedcontinuationindent {3ex } 
        \newcommand*\Wrappedafterbreak {\kern\Wrappedcontinuationindent\copy\Wrappedcontinuationbox} 
        % Take advantage of the already applied Pygments mark-up to insert 
        % potential linebreaks for TeX processing. 
        %        {, <, #, %, $, ' and ": go to next line. 
        %        _, }, ^, &, >, - and ~: stay at end of broken line. 
        % Use of \textquotesingle for straight quote. 
        \newcommand*\Wrappedbreaksatspecials {% 
            \def\PYGZus{\discretionary{\char`\_}{\Wrappedafterbreak}{\char`\_}}% 
            \def\PYGZob{\discretionary{}{\Wrappedafterbreak\char`\{}{\char`\{}}% 
            \def\PYGZcb{\discretionary{\char`\}}{\Wrappedafterbreak}{\char`\}}}% 
            \def\PYGZca{\discretionary{\char`\^}{\Wrappedafterbreak}{\char`\^}}% 
            \def\PYGZam{\discretionary{\char`\&}{\Wrappedafterbreak}{\char`\&}}% 
            \def\PYGZlt{\discretionary{}{\Wrappedafterbreak\char`\<}{\char`\<}}% 
            \def\PYGZgt{\discretionary{\char`\>}{\Wrappedafterbreak}{\char`\>}}% 
            \def\PYGZsh{\discretionary{}{\Wrappedafterbreak\char`\#}{\char`\#}}% 
            \def\PYGZpc{\discretionary{}{\Wrappedafterbreak\char`\%}{\char`\%}}% 
            \def\PYGZdl{\discretionary{}{\Wrappedafterbreak\char`\$}{\char`\$}}% 
            \def\PYGZhy{\discretionary{\char`\-}{\Wrappedafterbreak}{\char`\-}}% 
            \def\PYGZsq{\discretionary{}{\Wrappedafterbreak\textquotesingle}{\textquotesingle}}% 
            \def\PYGZdq{\discretionary{}{\Wrappedafterbreak\char`\"}{\char`\"}}% 
            \def\PYGZti{\discretionary{\char`\~}{\Wrappedafterbreak}{\char`\~}}% 
        } 
        % Some characters . , ; ? ! / are not pygmentized. 
        % This macro makes them "active" and they will insert potential linebreaks 
        \newcommand*\Wrappedbreaksatpunct {% 
            \lccode`\~`\.\lowercase{\def~}{\discretionary{\hbox{\char`\.}}{\Wrappedafterbreak}{\hbox{\char`\.}}}% 
            \lccode`\~`\,\lowercase{\def~}{\discretionary{\hbox{\char`\,}}{\Wrappedafterbreak}{\hbox{\char`\,}}}% 
            \lccode`\~`\;\lowercase{\def~}{\discretionary{\hbox{\char`\;}}{\Wrappedafterbreak}{\hbox{\char`\;}}}% 
            \lccode`\~`\:\lowercase{\def~}{\discretionary{\hbox{\char`\:}}{\Wrappedafterbreak}{\hbox{\char`\:}}}% 
            \lccode`\~`\?\lowercase{\def~}{\discretionary{\hbox{\char`\?}}{\Wrappedafterbreak}{\hbox{\char`\?}}}% 
            \lccode`\~`\!\lowercase{\def~}{\discretionary{\hbox{\char`\!}}{\Wrappedafterbreak}{\hbox{\char`\!}}}% 
            \lccode`\~`\/\lowercase{\def~}{\discretionary{\hbox{\char`\/}}{\Wrappedafterbreak}{\hbox{\char`\/}}}% 
            \catcode`\.\active
            \catcode`\,\active 
            \catcode`\;\active
            \catcode`\:\active
            \catcode`\?\active
            \catcode`\!\active
            \catcode`\/\active 
            \lccode`\~`\~ 	
        }
    \makeatother

    \let\OriginalVerbatim=\Verbatim
    \makeatletter
    \renewcommand{\Verbatim}[1][1]{%
        %\parskip\z@skip
        \sbox\Wrappedcontinuationbox {\Wrappedcontinuationsymbol}%
        \sbox\Wrappedvisiblespacebox {\FV@SetupFont\Wrappedvisiblespace}%
        \def\FancyVerbFormatLine ##1{\hsize\linewidth
            \vtop{\raggedright\hyphenpenalty\z@\exhyphenpenalty\z@
                \doublehyphendemerits\z@\finalhyphendemerits\z@
                \strut ##1\strut}%
        }%
        % If the linebreak is at a space, the latter will be displayed as visible
        % space at end of first line, and a continuation symbol starts next line.
        % Stretch/shrink are however usually zero for typewriter font.
        \def\FV@Space {%
            \nobreak\hskip\z@ plus\fontdimen3\font minus\fontdimen4\font
            \discretionary{\copy\Wrappedvisiblespacebox}{\Wrappedafterbreak}
            {\kern\fontdimen2\font}%
        }%
        
        % Allow breaks at special characters using \PYG... macros.
        \Wrappedbreaksatspecials
        % Breaks at punctuation characters . , ; ? ! and / need catcode=\active 	
        \OriginalVerbatim[#1,codes*=\Wrappedbreaksatpunct]%
    }
    \makeatother

    % Exact colors from NB
    \definecolor{incolor}{HTML}{303F9F}
    \definecolor{outcolor}{HTML}{D84315}
    \definecolor{cellborder}{HTML}{CFCFCF}
    \definecolor{cellbackground}{HTML}{F7F7F7}
    
    % prompt
    \makeatletter
    \newcommand{\boxspacing}{\kern\kvtcb@left@rule\kern\kvtcb@boxsep}
    \makeatother
    \newcommand{\prompt}[4]{
        \ttfamily\llap{{\color{#2}[#3]:\hspace{3pt}#4}}\vspace{-\baselineskip}
    }
    

    
    % Prevent overflowing lines due to hard-to-break entities
    \sloppy 
    % Setup hyperref package
    \hypersetup{
      breaklinks=true,  % so long urls are correctly broken across lines
      colorlinks=true,
      urlcolor=urlcolor,
      linkcolor=linkcolor,
      citecolor=citecolor,
      }
    % Slightly bigger margins than the latex defaults
    
    \geometry{verbose,tmargin=1in,bmargin=1in,lmargin=1in,rmargin=1in}
    
    

\begin{document}
    
    \maketitle
    
    

    
    \hypertarget{face-recognition}{%
\section{Face Recognition}\label{face-recognition}}

Welcome! In this assignment, you're going to build a face recognition
system. Many of the ideas presented here are from
\href{https://arxiv.org/pdf/1503.03832.pdf}{FaceNet}. In the lecture,
you also encountered
\href{https://research.fb.com/wp-content/uploads/2016/11/deepface-closing-the-gap-to-human-level-performance-in-face-verification.pdf}{DeepFace}.

Face recognition problems commonly fall into one of two categories:

\textbf{Face Verification} ``Is this the claimed person?'' For example,
at some airports, you can pass through customs by letting a system scan
your passport and then verifying that you (the person carrying the
passport) are the correct person. A mobile phone that unlocks using your
face is also using face verification. This is a 1:1 matching problem.

\textbf{Face Recognition} ``Who is this person?'' For example, the video
lecture showed a \href{https://www.youtube.com/watch?v=wr4rx0Spihs}{face
recognition video} of Baidu employees entering the office without
needing to otherwise identify themselves. This is a 1:K matching
problem.

FaceNet learns a neural network that encodes a face image into a vector
of 128 numbers. By comparing two such vectors, you can then determine if
two pictures are of the same person.

By the end of this assignment, you'll be able to:

\begin{itemize}
\tightlist
\item
  Differentiate between face recognition and face verification
\item
  Implement one-shot learning to solve a face recognition problem
\item
  Apply the triplet loss function to learn a network's parameters in the
  context of face recognition
\item
  Explain how to pose face recognition as a binary classification
  problem
\item
  Map face images into 128-dimensional encodings using a pretrained
  model
\item
  Perform face verification and face recognition with these encodings
\end{itemize}

\textbf{Channels-last notation}

For this assignment, you'll be using a pre-trained model which
represents ConvNet activations using a ``channels last'' convention, as
used during the lecture and in previous programming assignments.

In other words, a batch of images will be of shape
\((m, n_H, n_W, n_C)\).

    \hypertarget{table-of-contents}{%
\subsection{Table of Contents}\label{table-of-contents}}

\begin{itemize}
\tightlist
\item
  Section \ref{1}
\item
  Section \ref{2}
\item
  Section \ref{3}

  \begin{itemize}
  \tightlist
  \item
    Section \ref{3-1}
  \item
    Section \ref{3-2}

    \begin{itemize}
    \tightlist
    \item
      Section \ref{ex-1}
    \end{itemize}
  \end{itemize}
\item
  Section \ref{4}
\item
  Section \ref{5}

  \begin{itemize}
  \tightlist
  \item
    Section \ref{5-1}

    \begin{itemize}
    \tightlist
    \item
      Section \ref{ex-2}
    \end{itemize}
  \item
    Section \ref{5-2}

    \begin{itemize}
    \tightlist
    \item
      Section \ref{ex-3}
    \end{itemize}
  \end{itemize}
\item
  Section \ref{6}
\end{itemize}

    \#\# 1 - Packages

Go ahead and run the cell below to import the packages you'll need.

    \begin{tcolorbox}[breakable, size=fbox, boxrule=1pt, pad at break*=1mm,colback=cellbackground, colframe=cellborder]
\prompt{In}{incolor}{59}{\boxspacing}
\begin{Verbatim}[commandchars=\\\{\}]
\PY{k+kn}{from} \PY{n+nn}{tensorflow}\PY{n+nn}{.}\PY{n+nn}{keras}\PY{n+nn}{.}\PY{n+nn}{models} \PY{k+kn}{import} \PY{n}{Sequential}
\PY{k+kn}{from} \PY{n+nn}{tensorflow}\PY{n+nn}{.}\PY{n+nn}{keras}\PY{n+nn}{.}\PY{n+nn}{layers} \PY{k+kn}{import} \PY{n}{Conv2D}\PY{p}{,} \PY{n}{ZeroPadding2D}\PY{p}{,} \PY{n}{Activation}\PY{p}{,} \PY{n}{Input}\PY{p}{,} \PY{n}{concatenate}
\PY{k+kn}{from} \PY{n+nn}{tensorflow}\PY{n+nn}{.}\PY{n+nn}{keras}\PY{n+nn}{.}\PY{n+nn}{models} \PY{k+kn}{import} \PY{n}{Model}
\PY{k+kn}{from} \PY{n+nn}{tensorflow}\PY{n+nn}{.}\PY{n+nn}{keras}\PY{n+nn}{.}\PY{n+nn}{layers} \PY{k+kn}{import} \PY{n}{BatchNormalization}
\PY{k+kn}{from} \PY{n+nn}{tensorflow}\PY{n+nn}{.}\PY{n+nn}{keras}\PY{n+nn}{.}\PY{n+nn}{layers} \PY{k+kn}{import} \PY{n}{MaxPooling2D}\PY{p}{,} \PY{n}{AveragePooling2D}
\PY{k+kn}{from} \PY{n+nn}{tensorflow}\PY{n+nn}{.}\PY{n+nn}{keras}\PY{n+nn}{.}\PY{n+nn}{layers} \PY{k+kn}{import} \PY{n}{Concatenate}
\PY{k+kn}{from} \PY{n+nn}{tensorflow}\PY{n+nn}{.}\PY{n+nn}{keras}\PY{n+nn}{.}\PY{n+nn}{layers} \PY{k+kn}{import} \PY{n}{Lambda}\PY{p}{,} \PY{n}{Flatten}\PY{p}{,} \PY{n}{Dense}
\PY{k+kn}{from} \PY{n+nn}{tensorflow}\PY{n+nn}{.}\PY{n+nn}{keras}\PY{n+nn}{.}\PY{n+nn}{initializers} \PY{k+kn}{import} \PY{n}{glorot\PYZus{}uniform}
\PY{k+kn}{from} \PY{n+nn}{tensorflow}\PY{n+nn}{.}\PY{n+nn}{keras}\PY{n+nn}{.}\PY{n+nn}{layers} \PY{k+kn}{import} \PY{n}{Layer}
\PY{k+kn}{from} \PY{n+nn}{tensorflow}\PY{n+nn}{.}\PY{n+nn}{keras} \PY{k+kn}{import} \PY{n}{backend} \PY{k}{as} \PY{n}{K}
\PY{n}{K}\PY{o}{.}\PY{n}{set\PYZus{}image\PYZus{}data\PYZus{}format}\PY{p}{(}\PY{l+s+s1}{\PYZsq{}}\PY{l+s+s1}{channels\PYZus{}last}\PY{l+s+s1}{\PYZsq{}}\PY{p}{)}
\PY{k+kn}{import} \PY{n+nn}{os}
\PY{k+kn}{import} \PY{n+nn}{numpy} \PY{k}{as} \PY{n+nn}{np}
\PY{k+kn}{from} \PY{n+nn}{numpy} \PY{k+kn}{import} \PY{n}{genfromtxt}
\PY{k+kn}{import} \PY{n+nn}{pandas} \PY{k}{as} \PY{n+nn}{pd}
\PY{k+kn}{import} \PY{n+nn}{tensorflow} \PY{k}{as} \PY{n+nn}{tf}
\PY{k+kn}{import} \PY{n+nn}{PIL}

\PY{o}{\PYZpc{}}\PY{k}{matplotlib} inline
\PY{o}{\PYZpc{}}\PY{k}{load\PYZus{}ext} autoreload
\PY{o}{\PYZpc{}}\PY{k}{autoreload} 2
\end{Verbatim}
\end{tcolorbox}

    \begin{Verbatim}[commandchars=\\\{\}]
The autoreload extension is already loaded. To reload it, use:
  \%reload\_ext autoreload
    \end{Verbatim}

    \#\# 2 - Naive Face Verification

In Face Verification, you're given two images and you have to determine
if they are of the same person. The simplest way to do this is to
compare the two images pixel-by-pixel. If the distance between the raw
images is below a chosen threshold, it may be the same person!

Figure 1

Of course, this algorithm performs poorly, since the pixel values change
dramatically due to variations in lighting, orientation of the person's
face, minor changes in head position, and so on.

You'll see that rather than using the raw image, you can learn an
encoding, \(f(img)\).

By using an encoding for each image, an element-wise comparison produces
a more accurate judgement as to whether two pictures are of the same
person.

    \#\# 3 - Encoding Face Images into a 128-Dimensional Vector

\#\#\# 3.1 - Using a ConvNet to Compute Encodings

The FaceNet model takes a lot of data and a long time to train. So
following the common practice in applied deep learning, you'll load
weights that someone else has already trained. The network architecture
follows the Inception model from
\href{https://arxiv.org/abs/1409.4842}{Szegedy \emph{et al}..} An
Inception network implementation has been provided for you, and you can
find it in the file \texttt{inception\_blocks\_v2.py} to get a closer
look at how it is implemented.

\emph{Hot tip:} Go to ``File-\textgreater Open\ldots{}'' at the top of
this notebook. This opens the file directory that contains the
\texttt{.py} file).

The key things to be aware of are:

\begin{itemize}
\tightlist
\item
  This network uses 160x160 dimensional RGB images as its input.
  Specifically, a face image (or batch of \(m\) face images) as a tensor
  of shape \((m, n_H, n_W, n_C) = (m, 160, 160, 3)\)
\item
  The input images are originally of shape 96x96, thus, you need to
  scale them to 160x160. This is done in the
  \texttt{img\_to\_encoding()} function.
\item
  The output is a matrix of shape \((m, 128)\) that encodes each input
  face image into a 128-dimensional vector
\end{itemize}

Run the cell below to create the model for face images!

    \begin{tcolorbox}[breakable, size=fbox, boxrule=1pt, pad at break*=1mm,colback=cellbackground, colframe=cellborder]
\prompt{In}{incolor}{60}{\boxspacing}
\begin{Verbatim}[commandchars=\\\{\}]
\PY{k+kn}{from} \PY{n+nn}{tensorflow}\PY{n+nn}{.}\PY{n+nn}{keras}\PY{n+nn}{.}\PY{n+nn}{models} \PY{k+kn}{import} \PY{n}{model\PYZus{}from\PYZus{}json}

\PY{n}{json\PYZus{}file} \PY{o}{=} \PY{n+nb}{open}\PY{p}{(}\PY{l+s+s1}{\PYZsq{}}\PY{l+s+s1}{keras\PYZhy{}facenet\PYZhy{}h5/model.json}\PY{l+s+s1}{\PYZsq{}}\PY{p}{,} \PY{l+s+s1}{\PYZsq{}}\PY{l+s+s1}{r}\PY{l+s+s1}{\PYZsq{}}\PY{p}{)}
\PY{n}{loaded\PYZus{}model\PYZus{}json} \PY{o}{=} \PY{n}{json\PYZus{}file}\PY{o}{.}\PY{n}{read}\PY{p}{(}\PY{p}{)}
\PY{n}{json\PYZus{}file}\PY{o}{.}\PY{n}{close}\PY{p}{(}\PY{p}{)}
\PY{n}{model} \PY{o}{=} \PY{n}{model\PYZus{}from\PYZus{}json}\PY{p}{(}\PY{n}{loaded\PYZus{}model\PYZus{}json}\PY{p}{)}
\PY{n}{model}\PY{o}{.}\PY{n}{load\PYZus{}weights}\PY{p}{(}\PY{l+s+s1}{\PYZsq{}}\PY{l+s+s1}{keras\PYZhy{}facenet\PYZhy{}h5/model.h5}\PY{l+s+s1}{\PYZsq{}}\PY{p}{)}
\end{Verbatim}
\end{tcolorbox}

    Now summarize the input and output shapes:

    \begin{tcolorbox}[breakable, size=fbox, boxrule=1pt, pad at break*=1mm,colback=cellbackground, colframe=cellborder]
\prompt{In}{incolor}{61}{\boxspacing}
\begin{Verbatim}[commandchars=\\\{\}]
\PY{n+nb}{print}\PY{p}{(}\PY{n}{model}\PY{o}{.}\PY{n}{inputs}\PY{p}{)}
\PY{n+nb}{print}\PY{p}{(}\PY{n}{model}\PY{o}{.}\PY{n}{outputs}\PY{p}{)}
\end{Verbatim}
\end{tcolorbox}

    \begin{Verbatim}[commandchars=\\\{\}]
[<tf.Tensor 'input\_1\_2:0' shape=(None, 160, 160, 3) dtype=float32>]
[<tf.Tensor 'Bottleneck\_BatchNorm/batchnorm\_2/add\_1:0' shape=(None, 128)
dtype=float32>]
    \end{Verbatim}

    By using a 128-neuron fully connected layer as its last layer, the model
ensures that the output is an encoding vector of size 128. You then use
the encodings to compare two face images as follows:

Figure 2: By computing the distance between two encodings and
thresholding, you can determine if the two pictures represent the same
person

So, an encoding is a good one if:

\begin{itemize}
\tightlist
\item
  The encodings of two images of the same person are quite similar to
  each other.
\item
  The encodings of two images of different persons are very different.
\end{itemize}

The triplet loss function formalizes this, and tries to ``push'' the
encodings of two images of the same person (Anchor and Positive) closer
together, while ``pulling'' the encodings of two images of different
persons (Anchor, Negative) further apart.

Figure 3: In the next section, you'll call the pictures from left to
right: Anchor (A), Positive (P), Negative (N)

    \#\#\# 3.2 - The Triplet Loss

\textbf{Important Note}: Since you're using a pretrained model, you
won't actually need to implement the triplet loss function in this
assignment. \emph{However}, the triplet loss is the main ingredient of
the face recognition algorithm, and you'll need to know how to use it
for training your own FaceNet model, as well as other types of image
similarity problems. Therefore, you'll implement it below, for fun and
edification. :)

For an image \(x\), its encoding is denoted as \(f(x)\), where \(f\) is
the function computed by the neural network.

Training will use triplets of images \((A, P, N)\):

\begin{itemize}
\tightlist
\item
  A is an ``Anchor'' image--a picture of a person.
\item
  P is a ``Positive'' image--a picture of the same person as the Anchor
  image.
\item
  N is a ``Negative'' image--a picture of a different person than the
  Anchor image.
\end{itemize}

These triplets are picked from the training dataset.
\((A^{(i)}, P^{(i)}, N^{(i)})\) is used here to denote the \(i\)-th
training example.

You'd like to make sure that an image \(A^{(i)}\) of an individual is
closer to the Positive \(P^{(i)}\) than to the Negative image
\(N^{(i)}\)) by at least a margin \(\alpha\):

\[
|| f\left(A^{(i)}\right)-f\left(P^{(i)}\right)||_{2}^{2}+\alpha<|| f\left(A^{(i)}\right)-f\left(N^{(i)}\right)||_{2}^{2}
\]

You would thus like to minimize the following ``triplet cost'':

\[\mathcal{J} = \sum^{m}_{i=1} \large[ \small \underbrace{\mid \mid f(A^{(i)}) - f(P^{(i)}) \mid \mid_2^2}_\text{(1)} - \underbrace{\mid \mid f(A^{(i)}) - f(N^{(i)}) \mid \mid_2^2}_\text{(2)} + \alpha \large ] \small_+ \tag{3}\]
Here, the notation ``\([z]_+\)'' is used to denote \(max(z,0)\).

\textbf{Notes}:

\begin{itemize}
\tightlist
\item
  The term (1) is the squared distance between the anchor ``A'' and the
  positive ``P'' for a given triplet; you want this to be small.
\item
  The term (2) is the squared distance between the anchor ``A'' and the
  negative ``N'' for a given triplet, you want this to be relatively
  large. It has a minus sign preceding it because minimizing the
  negative of the term is the same as maximizing that term.
\item
  \(\alpha\) is called the margin. It's a hyperparameter that you pick
  manually. You'll use \(\alpha = 0.2\).
\end{itemize}

Most implementations also rescale the encoding vectors to haven L2 norm
equal to one (i.e., \(\mid \mid f(img)\mid \mid_2\)=1); you won't have
to worry about that in this assignment.

\#\#\# Exercise 1 - triplet\_loss

Implement the triplet loss as defined by formula (3). These are the 4
steps:

\begin{enumerate}
\def\labelenumi{\arabic{enumi}.}
\tightlist
\item
  Compute the distance between the encodings of ``anchor'' and
  ``positive'': \(\mid \mid f(A^{(i)}) - f(P^{(i)}) \mid \mid_2^2\)
\item
  Compute the distance between the encodings of ``anchor'' and
  ``negative'': \(\mid \mid f(A^{(i)}) - f(N^{(i)}) \mid \mid_2^2\)
\item
  Compute the formula per training example: \$ \mid \mid f(A\^{}\{(i)\})
  - f(P\^{}\{(i)\}) \mid \mid\_2\^{}2 - \mid \mid f(A\^{}\{(i)\}) -
  f(N\^{}\{(i)\}) \mid \mid\_2\^{}2 + \alpha\$
\item
  Compute the full formula by taking the max with zero and summing over
  the training
  examples:\[\mathcal{J} = \sum^{m}_{i=1} \large[ \small \mid \mid f(A^{(i)}) - f(P^{(i)}) \mid \mid_2^2 - \mid \mid f(A^{(i)}) - f(N^{(i)}) \mid \mid_2^2+ \alpha \large ] \small_+ \tag{3}\]
\end{enumerate}

\emph{Hints}:

\begin{itemize}
\item
  Useful functions: \texttt{tf.reduce\_sum()}, \texttt{tf.square()},
  \texttt{tf.subtract()}, \texttt{tf.add()}, \texttt{tf.maximum()}.
\item
  For steps 1 and 2, sum over the entries of
  \(\mid \mid f(A^{(i)}) - f(P^{(i)}) \mid \mid_2^2\) and
  \(\mid \mid f(A^{(i)}) - f(N^{(i)}) \mid \mid_2^2\).
\item
  For step 4, you will sum over the training examples.
\end{itemize}

\emph{Additional Hints}:

\begin{itemize}
\item
  Recall that the square of the L2 norm is the sum of the squared
  differences: \(||x - y||_{2}^{2} = \sum_{i=1}^{N}(x_{i} - y_{i})^{2}\)
\item
  Note that the anchor, positive and negative encodings are of shape
  (\emph{m},128), where \emph{m} is the number of training examples and
  128 is the number of elements used to encode a single example.
\item
  For steps 1 and 2, maintain the number of \emph{m} training examples
  and sum along the 128 values of each encoding. \texttt{tf.reduce\_sum}
  has an axis parameter. This chooses along which axis the sums are
  applied.
\item
  Note that one way to choose the last axis in a tensor is to use
  negative indexing (axis=-1).
\item
  In step 4, when summing over training examples, the result will be a
  single scalar value.
\item
  For \texttt{tf.reduce\_sum} to sum across all axes, keep the default
  value axis=None.
\end{itemize}

    \begin{tcolorbox}[breakable, size=fbox, boxrule=1pt, pad at break*=1mm,colback=cellbackground, colframe=cellborder]
\prompt{In}{incolor}{62}{\boxspacing}
\begin{Verbatim}[commandchars=\\\{\}]
\PY{c+c1}{\PYZsh{} UNQ\PYZus{}C1(UNIQUE CELL IDENTIFIER, DO NOT EDIT)}
\PY{c+c1}{\PYZsh{} GRADED FUNCTION: triplet\PYZus{}loss}

\PY{k}{def} \PY{n+nf}{triplet\PYZus{}loss}\PY{p}{(}\PY{n}{y\PYZus{}true}\PY{p}{,} \PY{n}{y\PYZus{}pred}\PY{p}{,} \PY{n}{alpha} \PY{o}{=} \PY{l+m+mf}{0.2}\PY{p}{)}\PY{p}{:}
    \PY{l+s+sd}{\PYZdq{}\PYZdq{}\PYZdq{}}
\PY{l+s+sd}{    Implementation of the triplet loss as defined by formula (3)}
\PY{l+s+sd}{    }
\PY{l+s+sd}{    Arguments:}
\PY{l+s+sd}{    y\PYZus{}true \PYZhy{}\PYZhy{} true labels, required when you define a loss in Keras, you don\PYZsq{}t need it in this function.}
\PY{l+s+sd}{    y\PYZus{}pred \PYZhy{}\PYZhy{} python list containing three objects:}
\PY{l+s+sd}{            anchor \PYZhy{}\PYZhy{} the encodings for the anchor images, of shape (None, 128)}
\PY{l+s+sd}{            positive \PYZhy{}\PYZhy{} the encodings for the positive images, of shape (None, 128)}
\PY{l+s+sd}{            negative \PYZhy{}\PYZhy{} the encodings for the negative images, of shape (None, 128)}
\PY{l+s+sd}{    }
\PY{l+s+sd}{    Returns:}
\PY{l+s+sd}{    loss \PYZhy{}\PYZhy{} real number, value of the loss}
\PY{l+s+sd}{    \PYZdq{}\PYZdq{}\PYZdq{}}
    
    \PY{n}{anchor}\PY{p}{,} \PY{n}{positive}\PY{p}{,} \PY{n}{negative} \PY{o}{=} \PY{n}{y\PYZus{}pred}\PY{p}{[}\PY{l+m+mi}{0}\PY{p}{]}\PY{p}{,} \PY{n}{y\PYZus{}pred}\PY{p}{[}\PY{l+m+mi}{1}\PY{p}{]}\PY{p}{,} \PY{n}{y\PYZus{}pred}\PY{p}{[}\PY{l+m+mi}{2}\PY{p}{]}
    
    \PY{c+c1}{\PYZsh{}\PYZsh{}\PYZsh{} START CODE HERE}
    \PY{c+c1}{\PYZsh{}(≈ 4 lines)}

    \PY{c+c1}{\PYZsh{} Step 1: Compute the (encoding) distance between the anchor and the positive}
    \PY{n}{pos\PYZus{}dist} \PY{o}{=} \PY{n}{tf}\PY{o}{.}\PY{n}{reduce\PYZus{}sum}\PY{p}{(}\PY{n}{tf}\PY{o}{.}\PY{n}{square}\PY{p}{(}\PY{n}{tf}\PY{o}{.}\PY{n}{subtract}\PY{p}{(}\PY{n}{anchor}\PY{p}{,} \PY{n}{positive}\PY{p}{)}\PY{p}{)}\PY{p}{,} \PY{n}{axis} \PY{o}{=} \PY{o}{\PYZhy{}}\PY{l+m+mi}{1}\PY{p}{)}
    \PY{c+c1}{\PYZsh{} Step 2: Compute the (encoding) distance between the anchor and the negative}
    \PY{n}{neg\PYZus{}dist} \PY{o}{=} \PY{n}{tf}\PY{o}{.}\PY{n}{reduce\PYZus{}sum}\PY{p}{(}\PY{n}{tf}\PY{o}{.}\PY{n}{square}\PY{p}{(}\PY{n}{tf}\PY{o}{.}\PY{n}{subtract}\PY{p}{(}\PY{n}{anchor}\PY{p}{,}\PY{n}{negative}\PY{p}{)}\PY{p}{)}\PY{p}{,} \PY{n}{axis} \PY{o}{=} \PY{o}{\PYZhy{}}\PY{l+m+mi}{1}\PY{p}{)}
    \PY{c+c1}{\PYZsh{} Step 3: subtract the two previous distances and add alpha.}
    \PY{n}{basic\PYZus{}loss} \PY{o}{=} \PY{n}{pos\PYZus{}dist} \PY{o}{\PYZhy{}} \PY{n}{neg\PYZus{}dist} \PY{o}{+} \PY{n}{alpha}
    \PY{c+c1}{\PYZsh{} Step 4: Take the maximum of basic\PYZus{}loss and 0.0. Sum over the training examples.}
    \PY{n}{loss} \PY{o}{=} \PY{n}{tf}\PY{o}{.}\PY{n}{reduce\PYZus{}sum}\PY{p}{(}\PY{n}{tf}\PY{o}{.}\PY{n}{maximum}\PY{p}{(}\PY{n}{basic\PYZus{}loss}\PY{p}{,} \PY{l+m+mi}{0}\PY{p}{)}\PY{p}{)}

    \PY{c+c1}{\PYZsh{}\PYZsh{}\PYZsh{} END CODE HERE}
    
    \PY{k}{return} \PY{n}{loss}
\end{Verbatim}
\end{tcolorbox}

    \begin{tcolorbox}[breakable, size=fbox, boxrule=1pt, pad at break*=1mm,colback=cellbackground, colframe=cellborder]
\prompt{In}{incolor}{63}{\boxspacing}
\begin{Verbatim}[commandchars=\\\{\}]
\PY{c+c1}{\PYZsh{} BEGIN UNIT TEST}
\PY{n}{tf}\PY{o}{.}\PY{n}{random}\PY{o}{.}\PY{n}{set\PYZus{}seed}\PY{p}{(}\PY{l+m+mi}{1}\PY{p}{)}
\PY{n}{y\PYZus{}true} \PY{o}{=} \PY{p}{(}\PY{k+kc}{None}\PY{p}{,} \PY{k+kc}{None}\PY{p}{,} \PY{k+kc}{None}\PY{p}{)} \PY{c+c1}{\PYZsh{} It is not used}
\PY{n}{y\PYZus{}pred} \PY{o}{=} \PY{p}{(}\PY{n}{tf}\PY{o}{.}\PY{n}{keras}\PY{o}{.}\PY{n}{backend}\PY{o}{.}\PY{n}{random\PYZus{}normal}\PY{p}{(}\PY{p}{[}\PY{l+m+mi}{3}\PY{p}{,} \PY{l+m+mi}{128}\PY{p}{]}\PY{p}{,} \PY{n}{mean}\PY{o}{=}\PY{l+m+mi}{6}\PY{p}{,} \PY{n}{stddev}\PY{o}{=}\PY{l+m+mf}{0.1}\PY{p}{,} \PY{n}{seed} \PY{o}{=} \PY{l+m+mi}{1}\PY{p}{)}\PY{p}{,}
          \PY{n}{tf}\PY{o}{.}\PY{n}{keras}\PY{o}{.}\PY{n}{backend}\PY{o}{.}\PY{n}{random\PYZus{}normal}\PY{p}{(}\PY{p}{[}\PY{l+m+mi}{3}\PY{p}{,} \PY{l+m+mi}{128}\PY{p}{]}\PY{p}{,} \PY{n}{mean}\PY{o}{=}\PY{l+m+mi}{1}\PY{p}{,} \PY{n}{stddev}\PY{o}{=}\PY{l+m+mi}{1}\PY{p}{,} \PY{n}{seed} \PY{o}{=} \PY{l+m+mi}{1}\PY{p}{)}\PY{p}{,}
          \PY{n}{tf}\PY{o}{.}\PY{n}{keras}\PY{o}{.}\PY{n}{backend}\PY{o}{.}\PY{n}{random\PYZus{}normal}\PY{p}{(}\PY{p}{[}\PY{l+m+mi}{3}\PY{p}{,} \PY{l+m+mi}{128}\PY{p}{]}\PY{p}{,} \PY{n}{mean}\PY{o}{=}\PY{l+m+mi}{3}\PY{p}{,} \PY{n}{stddev}\PY{o}{=}\PY{l+m+mi}{4}\PY{p}{,} \PY{n}{seed} \PY{o}{=} \PY{l+m+mi}{1}\PY{p}{)}\PY{p}{)}
\PY{n}{loss} \PY{o}{=} \PY{n}{triplet\PYZus{}loss}\PY{p}{(}\PY{n}{y\PYZus{}true}\PY{p}{,} \PY{n}{y\PYZus{}pred}\PY{p}{)}

\PY{k}{assert} \PY{n+nb}{type}\PY{p}{(}\PY{n}{loss}\PY{p}{)} \PY{o}{==} \PY{n}{tf}\PY{o}{.}\PY{n}{python}\PY{o}{.}\PY{n}{framework}\PY{o}{.}\PY{n}{ops}\PY{o}{.}\PY{n}{EagerTensor}\PY{p}{,} \PY{l+s+s2}{\PYZdq{}}\PY{l+s+s2}{Use tensorflow functions}\PY{l+s+s2}{\PYZdq{}}
\PY{n+nb}{print}\PY{p}{(}\PY{l+s+s2}{\PYZdq{}}\PY{l+s+s2}{loss = }\PY{l+s+s2}{\PYZdq{}} \PY{o}{+} \PY{n+nb}{str}\PY{p}{(}\PY{n}{loss}\PY{p}{)}\PY{p}{)}

\PY{n}{y\PYZus{}pred\PYZus{}perfect} \PY{o}{=} \PY{p}{(}\PY{p}{[}\PY{l+m+mf}{1.}\PY{p}{,} \PY{l+m+mf}{1.}\PY{p}{]}\PY{p}{,} \PY{p}{[}\PY{l+m+mf}{1.}\PY{p}{,} \PY{l+m+mf}{1.}\PY{p}{]}\PY{p}{,} \PY{p}{[}\PY{l+m+mf}{1.}\PY{p}{,} \PY{l+m+mf}{1.}\PY{p}{,}\PY{p}{]}\PY{p}{)}
\PY{n}{loss} \PY{o}{=} \PY{n}{triplet\PYZus{}loss}\PY{p}{(}\PY{n}{y\PYZus{}true}\PY{p}{,} \PY{n}{y\PYZus{}pred\PYZus{}perfect}\PY{p}{,} \PY{l+m+mi}{5}\PY{p}{)}
\PY{k}{assert} \PY{n}{loss} \PY{o}{==} \PY{l+m+mi}{5}\PY{p}{,} \PY{l+s+s2}{\PYZdq{}}\PY{l+s+s2}{Wrong value. Did you add the alpha to basic\PYZus{}loss?}\PY{l+s+s2}{\PYZdq{}}
\PY{n}{y\PYZus{}pred\PYZus{}perfect} \PY{o}{=} \PY{p}{(}\PY{p}{[}\PY{l+m+mf}{1.}\PY{p}{,} \PY{l+m+mf}{1.}\PY{p}{]}\PY{p}{,}\PY{p}{[}\PY{l+m+mf}{1.}\PY{p}{,} \PY{l+m+mf}{1.}\PY{p}{]}\PY{p}{,} \PY{p}{[}\PY{l+m+mf}{0.}\PY{p}{,} \PY{l+m+mf}{0.}\PY{p}{,}\PY{p}{]}\PY{p}{)}
\PY{n}{loss} \PY{o}{=} \PY{n}{triplet\PYZus{}loss}\PY{p}{(}\PY{n}{y\PYZus{}true}\PY{p}{,} \PY{n}{y\PYZus{}pred\PYZus{}perfect}\PY{p}{,} \PY{l+m+mi}{3}\PY{p}{)}
\PY{k}{assert} \PY{n}{loss} \PY{o}{==} \PY{l+m+mf}{1.}\PY{p}{,} \PY{l+s+s2}{\PYZdq{}}\PY{l+s+s2}{Wrong value. Check that pos\PYZus{}dist = 0 and neg\PYZus{}dist = 2 in this example}\PY{l+s+s2}{\PYZdq{}}
\PY{n}{y\PYZus{}pred\PYZus{}perfect} \PY{o}{=} \PY{p}{(}\PY{p}{[}\PY{l+m+mf}{1.}\PY{p}{,} \PY{l+m+mf}{1.}\PY{p}{]}\PY{p}{,}\PY{p}{[}\PY{l+m+mf}{0.}\PY{p}{,} \PY{l+m+mf}{0.}\PY{p}{]}\PY{p}{,} \PY{p}{[}\PY{l+m+mf}{1.}\PY{p}{,} \PY{l+m+mf}{1.}\PY{p}{,}\PY{p}{]}\PY{p}{)}
\PY{n}{loss} \PY{o}{=} \PY{n}{triplet\PYZus{}loss}\PY{p}{(}\PY{n}{y\PYZus{}true}\PY{p}{,} \PY{n}{y\PYZus{}pred\PYZus{}perfect}\PY{p}{,} \PY{l+m+mi}{0}\PY{p}{)}
\PY{k}{assert} \PY{n}{loss} \PY{o}{==} \PY{l+m+mf}{2.}\PY{p}{,} \PY{l+s+s2}{\PYZdq{}}\PY{l+s+s2}{Wrong value. Check that pos\PYZus{}dist = 2 and neg\PYZus{}dist = 0 in this example}\PY{l+s+s2}{\PYZdq{}}
\PY{n}{y\PYZus{}pred\PYZus{}perfect} \PY{o}{=} \PY{p}{(}\PY{p}{[}\PY{l+m+mf}{0.}\PY{p}{,} \PY{l+m+mf}{0.}\PY{p}{]}\PY{p}{,}\PY{p}{[}\PY{l+m+mf}{0.}\PY{p}{,} \PY{l+m+mf}{0.}\PY{p}{]}\PY{p}{,} \PY{p}{[}\PY{l+m+mf}{0.}\PY{p}{,} \PY{l+m+mf}{0.}\PY{p}{,}\PY{p}{]}\PY{p}{)}
\PY{n}{loss} \PY{o}{=} \PY{n}{triplet\PYZus{}loss}\PY{p}{(}\PY{n}{y\PYZus{}true}\PY{p}{,} \PY{n}{y\PYZus{}pred\PYZus{}perfect}\PY{p}{,} \PY{o}{\PYZhy{}}\PY{l+m+mi}{2}\PY{p}{)}
\PY{k}{assert} \PY{n}{loss} \PY{o}{==} \PY{l+m+mi}{0}\PY{p}{,} \PY{l+s+s2}{\PYZdq{}}\PY{l+s+s2}{Wrong value. Are you taking the maximum between basic\PYZus{}loss and 0?}\PY{l+s+s2}{\PYZdq{}}
\PY{n}{y\PYZus{}pred\PYZus{}perfect} \PY{o}{=} \PY{p}{(}\PY{p}{[}\PY{p}{[}\PY{l+m+mf}{1.}\PY{p}{,} \PY{l+m+mf}{0.}\PY{p}{]}\PY{p}{,} \PY{p}{[}\PY{l+m+mf}{1.}\PY{p}{,} \PY{l+m+mf}{0.}\PY{p}{]}\PY{p}{]}\PY{p}{,}\PY{p}{[}\PY{p}{[}\PY{l+m+mf}{1.}\PY{p}{,} \PY{l+m+mf}{0.}\PY{p}{]}\PY{p}{,} \PY{p}{[}\PY{l+m+mf}{1.}\PY{p}{,} \PY{l+m+mf}{0.}\PY{p}{]}\PY{p}{]}\PY{p}{,} \PY{p}{[}\PY{p}{[}\PY{l+m+mf}{0.}\PY{p}{,} \PY{l+m+mf}{1.}\PY{p}{]}\PY{p}{,} \PY{p}{[}\PY{l+m+mf}{0.}\PY{p}{,} \PY{l+m+mf}{1.}\PY{p}{]}\PY{p}{]}\PY{p}{)}
\PY{n}{loss} \PY{o}{=} \PY{n}{triplet\PYZus{}loss}\PY{p}{(}\PY{n}{y\PYZus{}true}\PY{p}{,} \PY{n}{y\PYZus{}pred\PYZus{}perfect}\PY{p}{,} \PY{l+m+mi}{3}\PY{p}{)}
\PY{k}{assert} \PY{n}{loss} \PY{o}{==} \PY{l+m+mf}{2.}\PY{p}{,} \PY{l+s+s2}{\PYZdq{}}\PY{l+s+s2}{Wrong value. Are you applying tf.reduce\PYZus{}sum to get the loss?}\PY{l+s+s2}{\PYZdq{}}
\PY{n}{y\PYZus{}pred\PYZus{}perfect} \PY{o}{=} \PY{p}{(}\PY{p}{[}\PY{p}{[}\PY{l+m+mf}{1.}\PY{p}{,} \PY{l+m+mf}{1.}\PY{p}{]}\PY{p}{,} \PY{p}{[}\PY{l+m+mf}{2.}\PY{p}{,} \PY{l+m+mf}{0.}\PY{p}{]}\PY{p}{]}\PY{p}{,} \PY{p}{[}\PY{p}{[}\PY{l+m+mf}{0.}\PY{p}{,} \PY{l+m+mf}{3.}\PY{p}{]}\PY{p}{,} \PY{p}{[}\PY{l+m+mf}{1.}\PY{p}{,} \PY{l+m+mf}{1.}\PY{p}{]}\PY{p}{]}\PY{p}{,} \PY{p}{[}\PY{p}{[}\PY{l+m+mf}{1.}\PY{p}{,} \PY{l+m+mf}{0.}\PY{p}{]}\PY{p}{,} \PY{p}{[}\PY{l+m+mf}{0.}\PY{p}{,} \PY{l+m+mf}{1.}\PY{p}{,}\PY{p}{]}\PY{p}{]}\PY{p}{)}
\PY{n}{loss} \PY{o}{=} \PY{n}{triplet\PYZus{}loss}\PY{p}{(}\PY{n}{y\PYZus{}true}\PY{p}{,} \PY{n}{y\PYZus{}pred\PYZus{}perfect}\PY{p}{,} \PY{l+m+mi}{1}\PY{p}{)}
\PY{k}{if} \PY{p}{(}\PY{n}{loss} \PY{o}{==} \PY{l+m+mf}{4.}\PY{p}{)}\PY{p}{:}
    \PY{k}{raise} \PY{n+ne}{Exception}\PY{p}{(}\PY{l+s+s1}{\PYZsq{}}\PY{l+s+s1}{Perhaps you are not using axis=\PYZhy{}1 in reduce\PYZus{}sum?}\PY{l+s+s1}{\PYZsq{}}\PY{p}{)}
\PY{k}{assert} \PY{n}{loss} \PY{o}{==} \PY{l+m+mi}{5}\PY{p}{,} \PY{l+s+s2}{\PYZdq{}}\PY{l+s+s2}{Wrong value. Check your implementation}\PY{l+s+s2}{\PYZdq{}}
\PY{c+c1}{\PYZsh{} END UNIT TEST}
\end{Verbatim}
\end{tcolorbox}

    \begin{Verbatim}[commandchars=\\\{\}]
loss = tf.Tensor(527.2598, shape=(), dtype=float32)
    \end{Verbatim}

    \textbf{Expected Output}:

loss

527.2598

    \#\# 4 - Loading the Pre-trained Model

FaceNet is trained by minimizing the triplet loss. But since training
requires a lot of data and a lot of computation, you won't train it from
scratch here. Instead, you'll load a previously trained model in the
following cell; which might take a couple of minutes to run.

    \begin{tcolorbox}[breakable, size=fbox, boxrule=1pt, pad at break*=1mm,colback=cellbackground, colframe=cellborder]
\prompt{In}{incolor}{36}{\boxspacing}
\begin{Verbatim}[commandchars=\\\{\}]
\PY{n}{FRmodel} \PY{o}{=} \PY{n}{model}
\end{Verbatim}
\end{tcolorbox}

    Here are some examples of distances between the encodings between three
individuals:

Figure 4: Example of distance outputs between three individuals'
encodings

Now use this model to perform face verification and face recognition!

    \#\# 5 - Applying the Model

You're building a system for an office building where the building
manager would like to offer facial recognition to allow the employees to
enter the building.

You'd like to build a face verification system that gives access to a
list of people. To be admitted, each person has to swipe an
identification card at the entrance. The face recognition system then
verifies that they are who they claim to be.

\#\#\# 5.1 - Face Verification

Now you'll build a database containing one encoding vector for each
person who is allowed to enter the office. To generate the encoding,
you'll use \texttt{img\_to\_encoding(image\_path,\ model)}, which runs
the forward propagation of the model on the specified image.

Run the following code to build the database (represented as a Python
dictionary). This database maps each person's name to a 128-dimensional
encoding of their face.

    \begin{tcolorbox}[breakable, size=fbox, boxrule=1pt, pad at break*=1mm,colback=cellbackground, colframe=cellborder]
\prompt{In}{incolor}{64}{\boxspacing}
\begin{Verbatim}[commandchars=\\\{\}]
\PY{c+c1}{\PYZsh{}tf.keras.backend.set\PYZus{}image\PYZus{}data\PYZus{}format(\PYZsq{}channels\PYZus{}last\PYZsq{})}
\PY{k}{def} \PY{n+nf}{img\PYZus{}to\PYZus{}encoding}\PY{p}{(}\PY{n}{image\PYZus{}path}\PY{p}{,} \PY{n}{model}\PY{p}{)}\PY{p}{:}
    \PY{n}{img} \PY{o}{=} \PY{n}{tf}\PY{o}{.}\PY{n}{keras}\PY{o}{.}\PY{n}{preprocessing}\PY{o}{.}\PY{n}{image}\PY{o}{.}\PY{n}{load\PYZus{}img}\PY{p}{(}\PY{n}{image\PYZus{}path}\PY{p}{,} \PY{n}{target\PYZus{}size}\PY{o}{=}\PY{p}{(}\PY{l+m+mi}{160}\PY{p}{,} \PY{l+m+mi}{160}\PY{p}{)}\PY{p}{)}
    \PY{n}{img} \PY{o}{=} \PY{n}{np}\PY{o}{.}\PY{n}{around}\PY{p}{(}\PY{n}{np}\PY{o}{.}\PY{n}{array}\PY{p}{(}\PY{n}{img}\PY{p}{)} \PY{o}{/} \PY{l+m+mf}{255.0}\PY{p}{,} \PY{n}{decimals}\PY{o}{=}\PY{l+m+mi}{12}\PY{p}{)}
    \PY{n}{x\PYZus{}train} \PY{o}{=} \PY{n}{np}\PY{o}{.}\PY{n}{expand\PYZus{}dims}\PY{p}{(}\PY{n}{img}\PY{p}{,} \PY{n}{axis}\PY{o}{=}\PY{l+m+mi}{0}\PY{p}{)}
    \PY{n}{embedding} \PY{o}{=} \PY{n}{model}\PY{o}{.}\PY{n}{predict\PYZus{}on\PYZus{}batch}\PY{p}{(}\PY{n}{x\PYZus{}train}\PY{p}{)}
    \PY{k}{return} \PY{n}{embedding} \PY{o}{/} \PY{n}{np}\PY{o}{.}\PY{n}{linalg}\PY{o}{.}\PY{n}{norm}\PY{p}{(}\PY{n}{embedding}\PY{p}{,} \PY{n+nb}{ord}\PY{o}{=}\PY{l+m+mi}{2}\PY{p}{)} \PY{c+c1}{\PYZsh{} normalized embedding is returned}
\end{Verbatim}
\end{tcolorbox}

    \begin{tcolorbox}[breakable, size=fbox, boxrule=1pt, pad at break*=1mm,colback=cellbackground, colframe=cellborder]
\prompt{In}{incolor}{65}{\boxspacing}
\begin{Verbatim}[commandchars=\\\{\}]
\PY{n}{database} \PY{o}{=} \PY{p}{\PYZob{}}\PY{p}{\PYZcb{}}
\PY{n}{database}\PY{p}{[}\PY{l+s+s2}{\PYZdq{}}\PY{l+s+s2}{danielle}\PY{l+s+s2}{\PYZdq{}}\PY{p}{]} \PY{o}{=} \PY{n}{img\PYZus{}to\PYZus{}encoding}\PY{p}{(}\PY{l+s+s2}{\PYZdq{}}\PY{l+s+s2}{images/danielle.png}\PY{l+s+s2}{\PYZdq{}}\PY{p}{,} \PY{n}{FRmodel}\PY{p}{)}
\PY{n}{database}\PY{p}{[}\PY{l+s+s2}{\PYZdq{}}\PY{l+s+s2}{younes}\PY{l+s+s2}{\PYZdq{}}\PY{p}{]} \PY{o}{=} \PY{n}{img\PYZus{}to\PYZus{}encoding}\PY{p}{(}\PY{l+s+s2}{\PYZdq{}}\PY{l+s+s2}{images/younes.jpg}\PY{l+s+s2}{\PYZdq{}}\PY{p}{,} \PY{n}{FRmodel}\PY{p}{)}
\PY{n}{database}\PY{p}{[}\PY{l+s+s2}{\PYZdq{}}\PY{l+s+s2}{tian}\PY{l+s+s2}{\PYZdq{}}\PY{p}{]} \PY{o}{=} \PY{n}{img\PYZus{}to\PYZus{}encoding}\PY{p}{(}\PY{l+s+s2}{\PYZdq{}}\PY{l+s+s2}{images/tian.jpg}\PY{l+s+s2}{\PYZdq{}}\PY{p}{,} \PY{n}{FRmodel}\PY{p}{)}
\PY{n}{database}\PY{p}{[}\PY{l+s+s2}{\PYZdq{}}\PY{l+s+s2}{andrew}\PY{l+s+s2}{\PYZdq{}}\PY{p}{]} \PY{o}{=} \PY{n}{img\PYZus{}to\PYZus{}encoding}\PY{p}{(}\PY{l+s+s2}{\PYZdq{}}\PY{l+s+s2}{images/andrew.jpg}\PY{l+s+s2}{\PYZdq{}}\PY{p}{,} \PY{n}{FRmodel}\PY{p}{)}
\PY{n}{database}\PY{p}{[}\PY{l+s+s2}{\PYZdq{}}\PY{l+s+s2}{kian}\PY{l+s+s2}{\PYZdq{}}\PY{p}{]} \PY{o}{=} \PY{n}{img\PYZus{}to\PYZus{}encoding}\PY{p}{(}\PY{l+s+s2}{\PYZdq{}}\PY{l+s+s2}{images/kian.jpg}\PY{l+s+s2}{\PYZdq{}}\PY{p}{,} \PY{n}{FRmodel}\PY{p}{)}
\PY{n}{database}\PY{p}{[}\PY{l+s+s2}{\PYZdq{}}\PY{l+s+s2}{dan}\PY{l+s+s2}{\PYZdq{}}\PY{p}{]} \PY{o}{=} \PY{n}{img\PYZus{}to\PYZus{}encoding}\PY{p}{(}\PY{l+s+s2}{\PYZdq{}}\PY{l+s+s2}{images/dan.jpg}\PY{l+s+s2}{\PYZdq{}}\PY{p}{,} \PY{n}{FRmodel}\PY{p}{)}
\PY{n}{database}\PY{p}{[}\PY{l+s+s2}{\PYZdq{}}\PY{l+s+s2}{sebastiano}\PY{l+s+s2}{\PYZdq{}}\PY{p}{]} \PY{o}{=} \PY{n}{img\PYZus{}to\PYZus{}encoding}\PY{p}{(}\PY{l+s+s2}{\PYZdq{}}\PY{l+s+s2}{images/sebastiano.jpg}\PY{l+s+s2}{\PYZdq{}}\PY{p}{,} \PY{n}{FRmodel}\PY{p}{)}
\PY{n}{database}\PY{p}{[}\PY{l+s+s2}{\PYZdq{}}\PY{l+s+s2}{bertrand}\PY{l+s+s2}{\PYZdq{}}\PY{p}{]} \PY{o}{=} \PY{n}{img\PYZus{}to\PYZus{}encoding}\PY{p}{(}\PY{l+s+s2}{\PYZdq{}}\PY{l+s+s2}{images/bertrand.jpg}\PY{l+s+s2}{\PYZdq{}}\PY{p}{,} \PY{n}{FRmodel}\PY{p}{)}
\PY{n}{database}\PY{p}{[}\PY{l+s+s2}{\PYZdq{}}\PY{l+s+s2}{kevin}\PY{l+s+s2}{\PYZdq{}}\PY{p}{]} \PY{o}{=} \PY{n}{img\PYZus{}to\PYZus{}encoding}\PY{p}{(}\PY{l+s+s2}{\PYZdq{}}\PY{l+s+s2}{images/kevin.jpg}\PY{l+s+s2}{\PYZdq{}}\PY{p}{,} \PY{n}{FRmodel}\PY{p}{)}
\PY{n}{database}\PY{p}{[}\PY{l+s+s2}{\PYZdq{}}\PY{l+s+s2}{felix}\PY{l+s+s2}{\PYZdq{}}\PY{p}{]} \PY{o}{=} \PY{n}{img\PYZus{}to\PYZus{}encoding}\PY{p}{(}\PY{l+s+s2}{\PYZdq{}}\PY{l+s+s2}{images/felix.jpg}\PY{l+s+s2}{\PYZdq{}}\PY{p}{,} \PY{n}{FRmodel}\PY{p}{)}
\PY{n}{database}\PY{p}{[}\PY{l+s+s2}{\PYZdq{}}\PY{l+s+s2}{benoit}\PY{l+s+s2}{\PYZdq{}}\PY{p}{]} \PY{o}{=} \PY{n}{img\PYZus{}to\PYZus{}encoding}\PY{p}{(}\PY{l+s+s2}{\PYZdq{}}\PY{l+s+s2}{images/benoit.jpg}\PY{l+s+s2}{\PYZdq{}}\PY{p}{,} \PY{n}{FRmodel}\PY{p}{)}
\PY{n}{database}\PY{p}{[}\PY{l+s+s2}{\PYZdq{}}\PY{l+s+s2}{arnaud}\PY{l+s+s2}{\PYZdq{}}\PY{p}{]} \PY{o}{=} \PY{n}{img\PYZus{}to\PYZus{}encoding}\PY{p}{(}\PY{l+s+s2}{\PYZdq{}}\PY{l+s+s2}{images/arnaud.jpg}\PY{l+s+s2}{\PYZdq{}}\PY{p}{,} \PY{n}{FRmodel}\PY{p}{)}
\end{Verbatim}
\end{tcolorbox}

    Load the images of Danielle and Kian:

    \begin{tcolorbox}[breakable, size=fbox, boxrule=1pt, pad at break*=1mm,colback=cellbackground, colframe=cellborder]
\prompt{In}{incolor}{66}{\boxspacing}
\begin{Verbatim}[commandchars=\\\{\}]
\PY{n}{danielle} \PY{o}{=} \PY{n}{tf}\PY{o}{.}\PY{n}{keras}\PY{o}{.}\PY{n}{preprocessing}\PY{o}{.}\PY{n}{image}\PY{o}{.}\PY{n}{load\PYZus{}img}\PY{p}{(}\PY{l+s+s2}{\PYZdq{}}\PY{l+s+s2}{images/danielle.png}\PY{l+s+s2}{\PYZdq{}}\PY{p}{,} \PY{n}{target\PYZus{}size}\PY{o}{=}\PY{p}{(}\PY{l+m+mi}{160}\PY{p}{,} \PY{l+m+mi}{160}\PY{p}{)}\PY{p}{)}
\PY{n}{kian} \PY{o}{=} \PY{n}{tf}\PY{o}{.}\PY{n}{keras}\PY{o}{.}\PY{n}{preprocessing}\PY{o}{.}\PY{n}{image}\PY{o}{.}\PY{n}{load\PYZus{}img}\PY{p}{(}\PY{l+s+s2}{\PYZdq{}}\PY{l+s+s2}{images/kian.jpg}\PY{l+s+s2}{\PYZdq{}}\PY{p}{,} \PY{n}{target\PYZus{}size}\PY{o}{=}\PY{p}{(}\PY{l+m+mi}{160}\PY{p}{,} \PY{l+m+mi}{160}\PY{p}{)}\PY{p}{)}
\end{Verbatim}
\end{tcolorbox}

    \begin{tcolorbox}[breakable, size=fbox, boxrule=1pt, pad at break*=1mm,colback=cellbackground, colframe=cellborder]
\prompt{In}{incolor}{67}{\boxspacing}
\begin{Verbatim}[commandchars=\\\{\}]
\PY{n}{np}\PY{o}{.}\PY{n}{around}\PY{p}{(}\PY{n}{np}\PY{o}{.}\PY{n}{array}\PY{p}{(}\PY{n}{kian}\PY{p}{)} \PY{o}{/} \PY{l+m+mf}{255.0}\PY{p}{,} \PY{n}{decimals}\PY{o}{=}\PY{l+m+mi}{12}\PY{p}{)}\PY{o}{.}\PY{n}{shape}
\end{Verbatim}
\end{tcolorbox}

            \begin{tcolorbox}[breakable, size=fbox, boxrule=.5pt, pad at break*=1mm, opacityfill=0]
\prompt{Out}{outcolor}{67}{\boxspacing}
\begin{Verbatim}[commandchars=\\\{\}]
(160, 160, 3)
\end{Verbatim}
\end{tcolorbox}
        
    \begin{tcolorbox}[breakable, size=fbox, boxrule=1pt, pad at break*=1mm,colback=cellbackground, colframe=cellborder]
\prompt{In}{incolor}{68}{\boxspacing}
\begin{Verbatim}[commandchars=\\\{\}]
\PY{n}{kian}
\end{Verbatim}
\end{tcolorbox}
 
            
\prompt{Out}{outcolor}{68}{}
    
    \begin{center}
    \adjustimage{max size={0.9\linewidth}{0.9\paperheight}}{output_23_0.png}
    \end{center}
    { \hspace*{\fill} \\}
    

    \begin{tcolorbox}[breakable, size=fbox, boxrule=1pt, pad at break*=1mm,colback=cellbackground, colframe=cellborder]
\prompt{In}{incolor}{69}{\boxspacing}
\begin{Verbatim}[commandchars=\\\{\}]
\PY{n}{np}\PY{o}{.}\PY{n}{around}\PY{p}{(}\PY{n}{np}\PY{o}{.}\PY{n}{array}\PY{p}{(}\PY{n}{danielle}\PY{p}{)} \PY{o}{/} \PY{l+m+mf}{255.0}\PY{p}{,} \PY{n}{decimals}\PY{o}{=}\PY{l+m+mi}{12}\PY{p}{)}\PY{o}{.}\PY{n}{shape}
\end{Verbatim}
\end{tcolorbox}

            \begin{tcolorbox}[breakable, size=fbox, boxrule=.5pt, pad at break*=1mm, opacityfill=0]
\prompt{Out}{outcolor}{69}{\boxspacing}
\begin{Verbatim}[commandchars=\\\{\}]
(160, 160, 3)
\end{Verbatim}
\end{tcolorbox}
        
    \begin{tcolorbox}[breakable, size=fbox, boxrule=1pt, pad at break*=1mm,colback=cellbackground, colframe=cellborder]
\prompt{In}{incolor}{70}{\boxspacing}
\begin{Verbatim}[commandchars=\\\{\}]
\PY{n}{danielle}
\end{Verbatim}
\end{tcolorbox}
 
            
\prompt{Out}{outcolor}{70}{}
    
    \begin{center}
    \adjustimage{max size={0.9\linewidth}{0.9\paperheight}}{output_25_0.png}
    \end{center}
    { \hspace*{\fill} \\}
    

    Now, when someone shows up at your front door and swipes their ID card
(thus giving you their name), you can look up their encoding in the
database, and use it to check if the person standing at the front door
matches the name on the ID.

\#\#\# Exercise 2 - verify

Implement the \texttt{verify()} function, which checks if the front-door
camera picture (\texttt{image\_path}) is actually the person called
``identity''. You will have to go through the following steps:

\begin{itemize}
\tightlist
\item
  Compute the encoding of the image from \texttt{image\_path}.
\item
  Compute the distance between this encoding and the encoding of the
  identity image stored in the database.
\item
  Open the door if the distance is less than 0.7, else do not open it.
\end{itemize}

As presented above, you should use the L2 distance
\texttt{np.linalg.norm}.

\textbf{Note}: In this implementation, compare the L2 distance, not the
square of the L2 distance, to the threshold 0.7.

\emph{Hints}:

\begin{itemize}
\tightlist
\item
  \texttt{identity} is a string that is also a key in the database
  dictionary.
\item
  \texttt{img\_to\_encoding} has two parameters: the image\_path and
  model.
\end{itemize}

    \begin{tcolorbox}[breakable, size=fbox, boxrule=1pt, pad at break*=1mm,colback=cellbackground, colframe=cellborder]
\prompt{In}{incolor}{72}{\boxspacing}
\begin{Verbatim}[commandchars=\\\{\}]
\PY{c+c1}{\PYZsh{} UNQ\PYZus{}C2(UNIQUE CELL IDENTIFIER, DO NOT EDIT)}
\PY{c+c1}{\PYZsh{} GRADED FUNCTION: verify}

\PY{k}{def} \PY{n+nf}{verify}\PY{p}{(}\PY{n}{image\PYZus{}path}\PY{p}{,} \PY{n}{identity}\PY{p}{,} \PY{n}{database}\PY{p}{,} \PY{n}{model}\PY{p}{)}\PY{p}{:}
    \PY{l+s+sd}{\PYZdq{}\PYZdq{}\PYZdq{}}
\PY{l+s+sd}{    Function that verifies if the person on the \PYZdq{}image\PYZus{}path\PYZdq{} image is \PYZdq{}identity\PYZdq{}.}
\PY{l+s+sd}{    }
\PY{l+s+sd}{    Arguments:}
\PY{l+s+sd}{        image\PYZus{}path \PYZhy{}\PYZhy{} path to an image}
\PY{l+s+sd}{        identity \PYZhy{}\PYZhy{} string, name of the person you\PYZsq{}d like to verify the identity. Has to be an employee who works in the office.}
\PY{l+s+sd}{        database \PYZhy{}\PYZhy{} python dictionary mapping names of allowed people\PYZsq{}s names (strings) to their encodings (vectors).}
\PY{l+s+sd}{        model \PYZhy{}\PYZhy{} your Inception model instance in Keras}
\PY{l+s+sd}{    }
\PY{l+s+sd}{    Returns:}
\PY{l+s+sd}{        dist \PYZhy{}\PYZhy{} distance between the image\PYZus{}path and the image of \PYZdq{}identity\PYZdq{} in the database.}
\PY{l+s+sd}{        door\PYZus{}open \PYZhy{}\PYZhy{} True, if the door should open. False otherwise.}
\PY{l+s+sd}{    \PYZdq{}\PYZdq{}\PYZdq{}}
    \PY{c+c1}{\PYZsh{}\PYZsh{}\PYZsh{} START CODE HERE}
    \PY{c+c1}{\PYZsh{} Step 1: Compute the encoding for the image. Use img\PYZus{}to\PYZus{}encoding() see example above. (≈ 1 line)}
    \PY{n}{encoding} \PY{o}{=} \PY{n}{img\PYZus{}to\PYZus{}encoding}\PY{p}{(}\PY{n}{image\PYZus{}path}\PY{p}{,} \PY{n}{model}\PY{p}{)}
    \PY{c+c1}{\PYZsh{} Step 2: Compute distance with identity\PYZsq{}s image (≈ 1 line)}
    \PY{n}{dist} \PY{o}{=} \PY{n}{np}\PY{o}{.}\PY{n}{linalg}\PY{o}{.}\PY{n}{norm}\PY{p}{(}\PY{n}{database}\PY{p}{[}\PY{n}{identity}\PY{p}{]} \PY{o}{\PYZhy{}} \PY{n}{encoding}\PY{p}{)}
    
    \PY{c+c1}{\PYZsh{} Step 3: Open the door if dist \PYZlt{} 0.7, else don\PYZsq{}t open (≈ 3 lines)}
    \PY{k}{if} \PY{n}{dist} \PY{o}{\PYZlt{}} \PY{l+m+mf}{0.7}\PY{p}{:}
        \PY{n+nb}{print}\PY{p}{(}\PY{l+s+s2}{\PYZdq{}}\PY{l+s+s2}{It}\PY{l+s+s2}{\PYZsq{}}\PY{l+s+s2}{s }\PY{l+s+s2}{\PYZdq{}} \PY{o}{+} \PY{n+nb}{str}\PY{p}{(}\PY{n}{identity}\PY{p}{)} \PY{o}{+} \PY{l+s+s2}{\PYZdq{}}\PY{l+s+s2}{, welcome in!}\PY{l+s+s2}{\PYZdq{}}\PY{p}{)}
        \PY{n}{door\PYZus{}open} \PY{o}{=} \PY{k+kc}{True}
    \PY{k}{else}\PY{p}{:}
        \PY{n+nb}{print}\PY{p}{(}\PY{l+s+s2}{\PYZdq{}}\PY{l+s+s2}{It}\PY{l+s+s2}{\PYZsq{}}\PY{l+s+s2}{s not }\PY{l+s+s2}{\PYZdq{}} \PY{o}{+} \PY{n+nb}{str}\PY{p}{(}\PY{n}{identity}\PY{p}{)} \PY{o}{+} \PY{l+s+s2}{\PYZdq{}}\PY{l+s+s2}{, please go away}\PY{l+s+s2}{\PYZdq{}}\PY{p}{)}
        \PY{n}{door\PYZus{}open} \PY{o}{=} \PY{k+kc}{False}
    \PY{c+c1}{\PYZsh{}\PYZsh{}\PYZsh{} END CODE HERE        }
    \PY{k}{return} \PY{n}{dist}\PY{p}{,} \PY{n}{door\PYZus{}open}
\end{Verbatim}
\end{tcolorbox}

    Younes is trying to enter the office and the camera takes a picture of
him (``images/camera\_0.jpg''). Let's run your verification algorithm on
this picture:

    \begin{tcolorbox}[breakable, size=fbox, boxrule=1pt, pad at break*=1mm,colback=cellbackground, colframe=cellborder]
\prompt{In}{incolor}{73}{\boxspacing}
\begin{Verbatim}[commandchars=\\\{\}]
\PY{c+c1}{\PYZsh{} BEGIN UNIT TEST}
\PY{k}{assert}\PY{p}{(}\PY{n}{np}\PY{o}{.}\PY{n}{allclose}\PY{p}{(}\PY{n}{verify}\PY{p}{(}\PY{l+s+s2}{\PYZdq{}}\PY{l+s+s2}{images/camera\PYZus{}1.jpg}\PY{l+s+s2}{\PYZdq{}}\PY{p}{,} \PY{l+s+s2}{\PYZdq{}}\PY{l+s+s2}{bertrand}\PY{l+s+s2}{\PYZdq{}}\PY{p}{,} \PY{n}{database}\PY{p}{,} \PY{n}{FRmodel}\PY{p}{)}\PY{p}{,} \PY{p}{(}\PY{l+m+mf}{0.54364836}\PY{p}{,} \PY{k+kc}{True}\PY{p}{)}\PY{p}{)}\PY{p}{)}
\PY{k}{assert}\PY{p}{(}\PY{n}{np}\PY{o}{.}\PY{n}{allclose}\PY{p}{(}\PY{n}{verify}\PY{p}{(}\PY{l+s+s2}{\PYZdq{}}\PY{l+s+s2}{images/camera\PYZus{}3.jpg}\PY{l+s+s2}{\PYZdq{}}\PY{p}{,} \PY{l+s+s2}{\PYZdq{}}\PY{l+s+s2}{bertrand}\PY{l+s+s2}{\PYZdq{}}\PY{p}{,} \PY{n}{database}\PY{p}{,} \PY{n}{FRmodel}\PY{p}{)}\PY{p}{,} \PY{p}{(}\PY{l+m+mf}{0.38616243}\PY{p}{,} \PY{k+kc}{True}\PY{p}{)}\PY{p}{)}\PY{p}{)}
\PY{k}{assert}\PY{p}{(}\PY{n}{np}\PY{o}{.}\PY{n}{allclose}\PY{p}{(}\PY{n}{verify}\PY{p}{(}\PY{l+s+s2}{\PYZdq{}}\PY{l+s+s2}{images/camera\PYZus{}1.jpg}\PY{l+s+s2}{\PYZdq{}}\PY{p}{,} \PY{l+s+s2}{\PYZdq{}}\PY{l+s+s2}{younes}\PY{l+s+s2}{\PYZdq{}}\PY{p}{,} \PY{n}{database}\PY{p}{,} \PY{n}{FRmodel}\PY{p}{)}\PY{p}{,} \PY{p}{(}\PY{l+m+mf}{1.3963861}\PY{p}{,} \PY{k+kc}{False}\PY{p}{)}\PY{p}{)}\PY{p}{)}
\PY{k}{assert}\PY{p}{(}\PY{n}{np}\PY{o}{.}\PY{n}{allclose}\PY{p}{(}\PY{n}{verify}\PY{p}{(}\PY{l+s+s2}{\PYZdq{}}\PY{l+s+s2}{images/camera\PYZus{}3.jpg}\PY{l+s+s2}{\PYZdq{}}\PY{p}{,} \PY{l+s+s2}{\PYZdq{}}\PY{l+s+s2}{younes}\PY{l+s+s2}{\PYZdq{}}\PY{p}{,} \PY{n}{database}\PY{p}{,} \PY{n}{FRmodel}\PY{p}{)}\PY{p}{,} \PY{p}{(}\PY{l+m+mf}{1.3872949}\PY{p}{,} \PY{k+kc}{False}\PY{p}{)}\PY{p}{)}\PY{p}{)}

\PY{n}{verify}\PY{p}{(}\PY{l+s+s2}{\PYZdq{}}\PY{l+s+s2}{images/camera\PYZus{}0.jpg}\PY{l+s+s2}{\PYZdq{}}\PY{p}{,} \PY{l+s+s2}{\PYZdq{}}\PY{l+s+s2}{younes}\PY{l+s+s2}{\PYZdq{}}\PY{p}{,} \PY{n}{database}\PY{p}{,} \PY{n}{FRmodel}\PY{p}{)}
\PY{c+c1}{\PYZsh{} END UNIT TEST}
\end{Verbatim}
\end{tcolorbox}

    \begin{Verbatim}[commandchars=\\\{\}]
It's bertrand, welcome in!
It's bertrand, welcome in!
It's not younes, please go away
It's not younes, please go away
It's younes, welcome in!
    \end{Verbatim}

            \begin{tcolorbox}[breakable, size=fbox, boxrule=.5pt, pad at break*=1mm, opacityfill=0]
\prompt{Out}{outcolor}{73}{\boxspacing}
\begin{Verbatim}[commandchars=\\\{\}]
(0.5992949, True)
\end{Verbatim}
\end{tcolorbox}
        
    \textbf{Expected Output}:

It's Younes, welcome in!

(0.5992946, True)

    Benoit, who does not work in the office, stole Kian's ID card and tried
to enter the office. Naughty Benoit! The camera took a picture of Benoit
("images/camera\_2.jpg).

Run the verification algorithm to check if Benoit can enter.

    \begin{tcolorbox}[breakable, size=fbox, boxrule=1pt, pad at break*=1mm,colback=cellbackground, colframe=cellborder]
\prompt{In}{incolor}{74}{\boxspacing}
\begin{Verbatim}[commandchars=\\\{\}]
\PY{n}{verify}\PY{p}{(}\PY{l+s+s2}{\PYZdq{}}\PY{l+s+s2}{images/camera\PYZus{}2.jpg}\PY{l+s+s2}{\PYZdq{}}\PY{p}{,} \PY{l+s+s2}{\PYZdq{}}\PY{l+s+s2}{kian}\PY{l+s+s2}{\PYZdq{}}\PY{p}{,} \PY{n}{database}\PY{p}{,} \PY{n}{FRmodel}\PY{p}{)}
\end{Verbatim}
\end{tcolorbox}

    \begin{Verbatim}[commandchars=\\\{\}]
It's not kian, please go away
    \end{Verbatim}

            \begin{tcolorbox}[breakable, size=fbox, boxrule=.5pt, pad at break*=1mm, opacityfill=0]
\prompt{Out}{outcolor}{74}{\boxspacing}
\begin{Verbatim}[commandchars=\\\{\}]
(1.0259346, False)
\end{Verbatim}
\end{tcolorbox}
        
    \textbf{Expected Output}:

It's not Kian, please go away

(1.0259346, False)

    \#\#\# 5.2 - Face Recognition

Your face verification system is mostly working. But since Kian got his
ID card stolen, when he came back to the office the next day he couldn't
get in!

To solve this, you'd like to change your face verification system to a
face recognition system. This way, no one has to carry an ID card
anymore. An authorized person can just walk up to the building, and the
door will unlock for them!

You'll implement a face recognition system that takes as input an image,
and figures out if it is one of the authorized persons (and if so, who).
Unlike the previous face verification system, you will no longer get a
person's name as one of the inputs.

\#\#\# Exercise 3 - who\_is\_it

Implement \texttt{who\_is\_it()} with the following steps:

\begin{itemize}
\tightlist
\item
  Compute the target encoding of the image from \texttt{image\_path}
\item
  Find the encoding from the database that has smallest distance with
  the target encoding.
\item
  Initialize the \texttt{min\_dist} variable to a large enough number
  (100). This helps you keep track of the closest encoding to the
  input's encoding.
\item
  Loop over the database dictionary's names and encodings. To loop use
  for (name, db\_enc) in \texttt{database.items()}.
\item
  Compute the L2 distance between the target ``encoding'' and the
  current ``encoding'' from the database. If this distance is less than
  the min\_dist, then set min\_dist to dist, and identity to name.
\end{itemize}

    \begin{tcolorbox}[breakable, size=fbox, boxrule=1pt, pad at break*=1mm,colback=cellbackground, colframe=cellborder]
\prompt{In}{incolor}{75}{\boxspacing}
\begin{Verbatim}[commandchars=\\\{\}]
\PY{c+c1}{\PYZsh{} UNQ\PYZus{}C3(UNIQUE CELL IDENTIFIER, DO NOT EDIT)}
\PY{c+c1}{\PYZsh{} GRADED FUNCTION: who\PYZus{}is\PYZus{}it}

\PY{k}{def} \PY{n+nf}{who\PYZus{}is\PYZus{}it}\PY{p}{(}\PY{n}{image\PYZus{}path}\PY{p}{,} \PY{n}{database}\PY{p}{,} \PY{n}{model}\PY{p}{)}\PY{p}{:}
    \PY{l+s+sd}{\PYZdq{}\PYZdq{}\PYZdq{}}
\PY{l+s+sd}{    Implements face recognition for the office by finding who is the person on the image\PYZus{}path image.}
\PY{l+s+sd}{    }
\PY{l+s+sd}{    Arguments:}
\PY{l+s+sd}{        image\PYZus{}path \PYZhy{}\PYZhy{} path to an image}
\PY{l+s+sd}{        database \PYZhy{}\PYZhy{} database containing image encodings along with the name of the person on the image}
\PY{l+s+sd}{        model \PYZhy{}\PYZhy{} your Inception model instance in Keras}
\PY{l+s+sd}{    }
\PY{l+s+sd}{    Returns:}
\PY{l+s+sd}{        min\PYZus{}dist \PYZhy{}\PYZhy{} the minimum distance between image\PYZus{}path encoding and the encodings from the database}
\PY{l+s+sd}{        identity \PYZhy{}\PYZhy{} string, the name prediction for the person on image\PYZus{}path}
\PY{l+s+sd}{    \PYZdq{}\PYZdq{}\PYZdq{}}
    
    \PY{c+c1}{\PYZsh{}\PYZsh{}\PYZsh{} START CODE HERE}

    \PY{c+c1}{\PYZsh{}\PYZsh{} Step 1: Compute the target \PYZdq{}encoding\PYZdq{} for the image. Use img\PYZus{}to\PYZus{}encoding() see example above. \PYZsh{}\PYZsh{} (≈ 1 line)}
    \PY{n}{encoding} \PY{o}{=}  \PY{n}{img\PYZus{}to\PYZus{}encoding}\PY{p}{(}\PY{n}{image\PYZus{}path}\PY{p}{,} \PY{n}{model}\PY{p}{)}
    
    \PY{c+c1}{\PYZsh{}\PYZsh{} Step 2: Find the closest encoding \PYZsh{}\PYZsh{}}
    
    \PY{c+c1}{\PYZsh{} Initialize \PYZdq{}min\PYZus{}dist\PYZdq{} to a large value, say 100 (≈1 line)}
    \PY{n}{min\PYZus{}dist} \PY{o}{=} \PY{l+m+mi}{1000}
    
    \PY{c+c1}{\PYZsh{} Loop over the database dictionary\PYZsq{}s names and encodings.}
    \PY{k}{for} \PY{p}{(}\PY{n}{name}\PY{p}{,} \PY{n}{db\PYZus{}enc}\PY{p}{)} \PY{o+ow}{in} \PY{n}{database}\PY{o}{.}\PY{n}{items}\PY{p}{(}\PY{p}{)}\PY{p}{:}
        
        \PY{c+c1}{\PYZsh{} Compute L2 distance between the target \PYZdq{}encoding\PYZdq{} and the current db\PYZus{}enc from the database. (≈ 1 line)}
        \PY{n}{dist} \PY{o}{=} \PY{n}{np}\PY{o}{.}\PY{n}{linalg}\PY{o}{.}\PY{n}{norm}\PY{p}{(}\PY{n}{encoding} \PY{o}{\PYZhy{}} \PY{n}{db\PYZus{}enc}\PY{p}{)}

        \PY{k}{if} \PY{n}{dist} \PY{o}{\PYZlt{}} \PY{n}{min\PYZus{}dist}\PY{p}{:}
            \PY{n}{min\PYZus{}dist} \PY{o}{=} \PY{n}{dist}
            \PY{n}{identity} \PY{o}{=} \PY{n}{name}
    \PY{c+c1}{\PYZsh{}\PYZsh{}\PYZsh{} END CODE HERE}
    
    \PY{k}{if} \PY{n}{min\PYZus{}dist} \PY{o}{\PYZgt{}} \PY{l+m+mf}{0.7}\PY{p}{:}
        \PY{n+nb}{print}\PY{p}{(}\PY{l+s+s2}{\PYZdq{}}\PY{l+s+s2}{Not in the database.}\PY{l+s+s2}{\PYZdq{}}\PY{p}{)}
    \PY{k}{else}\PY{p}{:}
        \PY{n+nb}{print} \PY{p}{(}\PY{l+s+s2}{\PYZdq{}}\PY{l+s+s2}{it}\PY{l+s+s2}{\PYZsq{}}\PY{l+s+s2}{s }\PY{l+s+s2}{\PYZdq{}} \PY{o}{+} \PY{n+nb}{str}\PY{p}{(}\PY{n}{identity}\PY{p}{)} \PY{o}{+} \PY{l+s+s2}{\PYZdq{}}\PY{l+s+s2}{, the distance is }\PY{l+s+s2}{\PYZdq{}} \PY{o}{+} \PY{n+nb}{str}\PY{p}{(}\PY{n}{min\PYZus{}dist}\PY{p}{)}\PY{p}{)}
        
    \PY{k}{return} \PY{n}{min\PYZus{}dist}\PY{p}{,} \PY{n}{identity}
\end{Verbatim}
\end{tcolorbox}

    Younes is at the front door and the camera takes a picture of him
(``images/camera\_0.jpg''). Let's see if your \texttt{who\_it\_is()}
algorithm identifies Younes.

    \begin{tcolorbox}[breakable, size=fbox, boxrule=1pt, pad at break*=1mm,colback=cellbackground, colframe=cellborder]
\prompt{In}{incolor}{76}{\boxspacing}
\begin{Verbatim}[commandchars=\\\{\}]
\PY{c+c1}{\PYZsh{} BEGIN UNIT TEST}
\PY{c+c1}{\PYZsh{} Test 1 with Younes pictures }
\PY{n}{who\PYZus{}is\PYZus{}it}\PY{p}{(}\PY{l+s+s2}{\PYZdq{}}\PY{l+s+s2}{images/camera\PYZus{}0.jpg}\PY{l+s+s2}{\PYZdq{}}\PY{p}{,} \PY{n}{database}\PY{p}{,} \PY{n}{FRmodel}\PY{p}{)}

\PY{c+c1}{\PYZsh{} Test 2 with Younes pictures }
\PY{n}{test1} \PY{o}{=} \PY{n}{who\PYZus{}is\PYZus{}it}\PY{p}{(}\PY{l+s+s2}{\PYZdq{}}\PY{l+s+s2}{images/camera\PYZus{}0.jpg}\PY{l+s+s2}{\PYZdq{}}\PY{p}{,} \PY{n}{database}\PY{p}{,} \PY{n}{FRmodel}\PY{p}{)}
\PY{k}{assert} \PY{n}{np}\PY{o}{.}\PY{n}{isclose}\PY{p}{(}\PY{n}{test1}\PY{p}{[}\PY{l+m+mi}{0}\PY{p}{]}\PY{p}{,} \PY{l+m+mf}{0.5992946}\PY{p}{)}
\PY{k}{assert} \PY{n}{test1}\PY{p}{[}\PY{l+m+mi}{1}\PY{p}{]} \PY{o}{==} \PY{l+s+s1}{\PYZsq{}}\PY{l+s+s1}{younes}\PY{l+s+s1}{\PYZsq{}}

\PY{c+c1}{\PYZsh{} Test 3 with Younes pictures }
\PY{n}{test2} \PY{o}{=} \PY{n}{who\PYZus{}is\PYZus{}it}\PY{p}{(}\PY{l+s+s2}{\PYZdq{}}\PY{l+s+s2}{images/younes.jpg}\PY{l+s+s2}{\PYZdq{}}\PY{p}{,} \PY{n}{database}\PY{p}{,} \PY{n}{FRmodel}\PY{p}{)}
\PY{k}{assert} \PY{n}{np}\PY{o}{.}\PY{n}{isclose}\PY{p}{(}\PY{n}{test2}\PY{p}{[}\PY{l+m+mi}{0}\PY{p}{]}\PY{p}{,} \PY{l+m+mf}{0.0}\PY{p}{)}
\PY{k}{assert} \PY{n}{test2}\PY{p}{[}\PY{l+m+mi}{1}\PY{p}{]} \PY{o}{==} \PY{l+s+s1}{\PYZsq{}}\PY{l+s+s1}{younes}\PY{l+s+s1}{\PYZsq{}}
\PY{c+c1}{\PYZsh{} END UNIT TEST}
\end{Verbatim}
\end{tcolorbox}

    \begin{Verbatim}[commandchars=\\\{\}]
it's younes, the distance is 0.5992949
it's younes, the distance is 0.5992949
it's younes, the distance is 0.0
    \end{Verbatim}

    \textbf{Expected Output}:

it's Younes, the distance is 0.5992946

(0.5992946, `younes')

You can change ``camera\_0.jpg'' (picture of Younes) to
``camera\_1.jpg'' (picture of Bertrand) and see the result.

    \textbf{Congratulations}! You've completed this assignment, and your
face recognition system is working well! It not only lets in authorized
persons, but now people don't need to carry an ID card around anymore!

You've now seen how a state-of-the-art face recognition system works,
and can describe the difference between face recognition and face
verification. Here's a quick recap of what you've accomplished:

\begin{itemize}
\tightlist
\item
  Posed face recognition as a binary classification problem
\item
  Implemented one-shot learning for a face recognition problem
\item
  Applied the triplet loss function to learn a network's parameters in
  the context of face recognition
\item
  Mapped face images into 128-dimensional encodings using a pretrained
  model
\item
  Performed face verification and face recognition with these encodings
\end{itemize}

Great work!

    \textbf{What you should remember}:

\begin{itemize}
\item
  Face verification solves an easier 1:1 matching problem; face
  recognition addresses a harder 1:K matching problem.
\item
  Triplet loss is an effective loss function for training a neural
  network to learn an encoding of a face image.
\item
  The same encoding can be used for verification and recognition.
  Measuring distances between two images' encodings allows you to
  determine whether they are pictures of the same person.
\end{itemize}

    \textbf{Ways to improve your facial recognition model}:

Although you won't implement these here, here are some ways to further
improve the algorithm:

\begin{itemize}
\item
  Put more images of each person (under different lighting conditions,
  taken on different days, etc.) into the database. Then, given a new
  image, compare the new face to multiple pictures of the person. This
  would increase accuracy.
\item
  Crop the images to contain just the face, and less of the ``border''
  region around the face. This preprocessing removes some of the
  irrelevant pixels around the face, and also makes the algorithm more
  robust.
\end{itemize}

    \#\# 6 - References 1. Florian Schroff, Dmitry Kalenichenko, James
Philbin (2015). \href{https://arxiv.org/pdf/1503.03832.pdf}{FaceNet: A
Unified Embedding for Face Recognition and Clustering}

\begin{enumerate}
\def\labelenumi{\arabic{enumi}.}
\setcounter{enumi}{1}
\item
  Yaniv Taigman, Ming Yang, Marc'Aurelio Ranzato, Lior Wolf (2014).
  \href{https://research.fb.com/wp-content/uploads/2016/11/deepface-closing-the-gap-to-human-level-performance-in-face-verification.pdf}{DeepFace:
  Closing the gap to human-level performance in face verification}
\item
  This implementation also took a lot of inspiration from the official
  FaceNet github repository: https://github.com/davidsandberg/facenet
\item
  Further inspiration was found here:
  https://machinelearningmastery.com/how-to-develop-a-face-recognition-system-using-facenet-in-keras-and-an-svm-classifier/
\item
  And here:
  https://github.com/nyoki-mtl/keras-facenet/blob/master/notebook/tf\_to\_keras.ipynb
\end{enumerate}


    % Add a bibliography block to the postdoc
    
    
    
\end{document}
